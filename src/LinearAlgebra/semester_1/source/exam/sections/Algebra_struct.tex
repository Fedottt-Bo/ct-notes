\subsection{Операции, группа, кольцо, поле}
\subsubsection{Законы композиции.}

$f: A \times B \rightarrow C$ - функция отображения.

$\forall (a,b): a \in A, b \in B: \exists! c \in C$ --- \deff{закон внешней контрапозиции.}

$f: A \times A \rightarrow A$ --- \deff{закон внутренней контрапозиции} или алгебраическая операция.

\subsubsection{Ассоциативность, коммутативность алгебраических операций.}

Возьмем операцию $*: A \times A \rightarrow A$:

$a * b = b * a$ --- \deff{коммутативность}.

$a * (b *c) = (a * b) *c$ --- \deff{ассоциативность}.


\subsubsection{Алгебраическая структура, группа, кольцо, поле. Свойства.}

\deff{Алгебраическая структура} --- множество с набором $\Omega$ --- операция и отношений на ней, с некоторой системой аксиом. Обозначают $(A, \Omega)$

\deff{Группа} (A, \{+\}):

\begin{enumerate}
    \item $a+(b+c) = (a+b)+c$ --- ассоциативность.
    \item $\exists 0:\forall a: a + 0 = 0+ a =a$.
    \item $\exists$  обратного: $\forall a: \exists (-a): a+ (-a) = 0$.
\end{enumerate}

Если группа обладает еще и коммутативностью, то такая группа называется \deff{абелевой}.

\deff{Кольцо} (A, \{+, $\cdot$\}):

\begin{enumerate}
    \item Абелева групппа по сложению.
    \item $a \cdot (b+c) = a\cdot b + a\cdot c$ --- левая дистрибутивность.
    \item $(b+c)\cdot a  = b \cdot a + c \cdot a$  --- правая дистрибутивность.
    \item $a \cdot (b\cdot c) = (a\cdot b)\cdot c$ --- ассоциативность умножения
\end{enumerate}

\deff{Поле}  (A, \{+, $\cdot$\}):
\begin{enumerate}
 \item Абелева групппа по сложению.
 \item Абелева групппа по умножению.
 \item $a(b+c) = ab+ac$ --- дистрибутивность. 
\end{enumerate}

Свойства кольца:

\begin{enumerate}
    \item $0 \cdot a = 0$
    \item $a+x = a+ y \rightarrow x=y$
    \item $a + x = b$ имеет единственное решение
    \item $0$ --- единственнен.
    \item $1$ --- единственна в кольце с единицой.
\end{enumerate}

\subsection{Линейное пространство, алгебра, свойства.}

$K - $ поле, $V$ - множество. $+: V \times V \rightarrow V$, $\cdot: K\times V \rightarrow V$. Если все, что сказано ниже выполнено  $\forall \phi, \lambda \in K, a,b \in V$.

\begin{enumerate}
    \item аксиомы
    \item аббелевой 
    \item для V
    \item по сложению.
    \item $ \phi(\lambda(a)) = \lambda(\phi(a))$.
    \item $\lambda (a+b) = \lambda a + \lambda b$.
    \item $a ( \phi + \lambda) = a\phi + a \lambda $.
    \item $\exists 1: a \cdot 1 =a$.
\end{enumerate}

То тогда такую систему называют \deff{линейным пространством}.

Если добавить еще одну операцию $\times: V \times V \rightarrow V$.

\begin{enumerate}
    \item $(a+b)\times c = a \times c + b \times c $

    $c\times (a+b) = c \times a + c \times b$

    \item $\lambda (a \times b) = (\lambda a )\times b = a \times (\lambda b)$
\end{enumerate}

То такую штучку называют \deff{линейной алгеброй}.

\begin{enumerate}
    \item добавим коммутативность $\times$ --- коммутативная алгебра.

    \item добавим ассоциативность $\times$ --- ассоциативная алгебра.

    \item добавим единицу --- унарная алгебра.

    \item добавим обратное --- алгебра с делением
\end{enumerate}

\subsection{Нормированные линейные пространства и алгебры.}

\deff{Нормированное пространство} --- линейное пространство над $\mathbb{R}$ с нормой.

\deff{Норма} $||\cdot||:V \rightarrow \mathbb{R}$, удовлетворящее:

\begin{enumerate}
    \item $||x|| + ||y||\leq ||x+y||$.
    \item $||x||\geq 0$, причем $||x||=0 \Leftrightarrow$ $x = 0$.
    \item  $||\alpha x|| = |\alpha|||x||$.
\end{enumerate}

Алгебра называется \deff{нормированной}, если существует норма согласованная с умножением:

$||ab||\leq ||a|| \cdot ||b||$.

\subsection{Отношение эквивалентности, фактор-структуры.}

         
        
        \deff{Бинарное отношение} $\sim$ на множестве $X$ является \deff{отношением эквивалентности}, если оно
        \begin{itemize}
            \item Рефлексивно: $\forall x\in X~x\sim x$.
            \item Симметрично: $\forall x,y\in X~x\sim y\leftrightarrow y\sim x$.
            \item Транзитивно: $\forall x,y,z\in X~x\sim y\land y\sim z\rightarrow x\sim z$.
        \end{itemize}
         Если $\sim$ --- бинарное отношение на $X$, то множества $M_a=\{x\in X\mid x\sim a\}$ называются классами эквивалентности , а множество $X/\sim=\{M_a\mid a\in X\}$ --- \deff{фактормножеством} (или {факторпространством}) $X$ по $\sim$.
         
     \thmm{Свойства классов эквивалентности.}
        \begin{enumerate}
            \item $\forall a\in X~M_a\neq\varnothing$.
            \item $\forall a,b\in X$ выполнено либо $M_a=M_b$, либо $M_a\cap M_b=\varnothing$.
            \item $\bigcup\limits_{a\in X}M_a=X$.
        \end{enumerate}

          Если у нас есть множество $X$, а $M$ --- какое-то множество, состоящее из непустых взаимно непересекающихся подмножеств $X$, в объединении дающих $X$. Тогда $M$ называется \deff{разбиением} $X$.


          Любое разбиение $X$ является факторпространством $X$ по некоторому отношению эквивалентности. Доказательство этого тривиально, если вы представите отношения как ребра в графе, а классы эквивалентности - компоненты
  

    
