\subsection{Введение в JavaScript}
\epigraph{Если вы на практике хотите сравнивать NaN и undefined, то очень жаль, что вам до этого кто-то голову не открутил.}{Георгий Корнеев}
\subsubsection{О языке:}
\begin{itemize}
    \item Изначально создан для добавления интерактивности веб-страницам. Сейчас используется гораздо шире: для разработки веб-приложений (frontend и backend), мобильных приложений, десктопных приложений, игр и многого другого.
    \item Официально язык стандартизирован как ECMAScript (ECMA-262). JavaScript - это наиболее распространённая реализация стандарта ECMAScript.
    \item Таким образом, строго говоря, "ECMAScript" - это спецификация языка, а "JavaScript" - конкретная реализация этой спецификации. Но на практике эти термины часто используются как взаимозаменяемые.
\end{itemize}

\subsubsection{Как пишется код? Основные вещи.}

\textbf{Типы данных и объявление переменных:}

Переменные объявляются с помощью ключевого слова let:\begin{minted}{javascript}
let v = 10;
\end{minted}
С помощью typeof мы можем более точно понять, что за тип:
\begin{minted}{javascript}
>> a = 3
>> typeof(a)
"number"
>> a = hello
>> typeof(a)
"string"
>> a = [1, 2, 3]
>> typeof(a)
"object"
\end{minted}
На следующей лекции мы узнаем, что такое object, пока что забьем.

Так же в Javascript есть не только null, но и undefined. Смысл у этого такой, что:
\begin{minted}{javascript}
let a = null; у переменной есть значение и это ничего  === null
let b; у переменной нет значения === undefined 
\end{minted}

\textbf{Веселые места в javascript (оператор равенства):}

\epigraph{Не надо интерпретировать массивы как числа и у вас все будет хорошо!}{Георгий Корнеев}

Давайте рассмотрим некоторые \deff{приколы}:
\begin{minted}{javascript}
>> "1" == "1"
true
\end{minted}
Ну это звучит максимально логично, пока мы не узнаем, что:
\begin{minted}{javascript}
>> "1" == 1
true
>> "1" == 1.0
true
>> "1.0" == 1
true
\end{minted}
Бывают более интересные вещи:
\begin{minted}{javascript}
>> 0 == []
true
>> undefined  == null
true
>> 10 == [10]
true
\end{minted}

Для этого в JavaScript есть 2 типа сравнения: приводящее (`==`) и неприводящее (`===`). Приводящее приводит данные к одному типу, а затем сравнивает, а неприводящее выдаёт `false`, если типы не равны.

\textbf{Типизация.}

Во-первых стоит разделять 2 вида типизации:
\begin{enumerate}
    \item \deff{статическую} - во время компиляции.
    \item \deff{динамическую} - во время исполнения.
\end{enumerate}

Статической типизации в JavaScript \deff{нет}. В Javascript слабая динамическая типизация.

\textbf{Массивы.}

Посмотрим на примерах, они крайне подробно описывают происходящее:
\begin{minted}{javascript}
>> a = [10, 20, 30];
>> a.push(40,50);
>> dump(a);
length: 5, elements: [10, 20, 30 , 40, 50]
>> a.pop();
50
>> a.pop();
40
>> a.unshift(100); - вставить в начало, никто не знает за сколько это работает. Приличные люди не задают такие вопросы, особенно JavaScript-у
>> a.shift();
100
>> a[10] = 10;
>> dump(a);
length: 11, elements: [10, 20, 30, , , , , , , , 10]
>> a[5]
undefined
>> a["hello"] = "world"
>> a[-1] = "???"
>> dump(a)
length: 11, elements: [10, 20, 30, , , , , , , , 10], other{-1 = ???, hello = world}
>> as[as]  = as
length: 11, elements: [10, 20, 30, , , , , , , , 10], other{-1 = ???, hello = world, 10, 20, 30, , , , , , , , 10 = 10, 20, 30, , , , , , , , 10 }
>> a.pop();
10
>> a.dump();
length: 11, elements: [10, 20, 30, , , , , , , , 10], other{-1 = ???, hello = world, 10, 20, 30, , , , , , , , 10 = 10, 20, 30, , , , , , ,}
\end{minted}
Как видно массивы в JavaScript работают превосходно!

\textbf{Константы}

\begin{minted}{javascript}
>> const c = 10
>> c = 20
Uncaught TypeError: Assignment to constant variable.
\end{minted}
Константы нельзя обновлять. Но например, если это будет объект, то внутри него можно будет менять значения.

\subsubsection{Функции.}


\textbf{Функции - это переменные:}

\begin{minted}{javascript}
let dumpArgs = function(){
    println(typeof(arguments));
    println(arguments.constructor.name);
}
>> dumpArgs(10, "hello", null);
object
Object  --- иногда Arguments
\end{minted}
Вот так создаются функции. Arguments ведет себя как массив:

\begin{minted}{javascript}
dumpArgs = function() {
    for (const v of arguments) {
        println(v);
    }
}
>> dumpArgs(1, 2, "hello", null)
1
2
hello
null
\end{minted}
Вообще у нас const v - неизменняя переменная. Слово переменная говорит о том, что она меняется. А модифактор const говорит, что он не может. Задумайтесь :)
\begin{minted}{javascript}
sum = function() {
    let s = 0;
    for (const v of arguments) {
        s += v;
    }
    return s;
}
>> sum(1, 2, 3)
6
\end{minted}
У javascript есть:
\begin{minted}{javascript}
use strict
\end{minted}
Он запрещает производить некоторые совсем извращенские штуки.


Ещё функции можно объявлять так:
\begin{minted}{javascript}
function min(a, b) {
    return a < b ? a : b;
}
>> min(3, 4)
3
\end{minted}
Но это не означает, что мы не можем вызывать функцию с другим количеством аргументов:
\begin{minted}{javascript}
>> min(4)
undefined
>> min(1, 2, -10)
1
\end{minted}
Формально он просто именует первые $n$ аргументов (в данном примере 2).


Можно положить все остальные аргументы в массив:
\begin{minted}{javascript}
 function min(first, ...rest) {
    for (const v of rest) {
        if (v < first) {
            first = v;
        }
    }
    return first;
}
>> min(1, 2, 3, -10, 5)
-10
\end{minted}
После rest параметра ничего быть не может.

\textbf{Второй вид функций ---стрелочные функции.}
\begin{minted}{javascript}
let min = (first, ...rest) =>{
    for (const v of rest) {
        if (v < first) {
            first = v;
        }
    }
    return first;
}
\end{minted}
Существенное отличие --- все аргументы именованные.

Можно писать еще:
\begin{minted}{javascript}
const add = (a, b) => a + b.
\end{minted}

\textbf{Функции в аргументы к функциям. Легко!}

Функции можно передавать в аргументы другим функциям и возвращать как результат! Такие функции называются \deff{функциями высшего порядка}: 
\begin{minted}{javascript}
minByComparator = (compare, init = Infinity) => {
    return (...args) => {
        let result = init;
        for (const v of args) {
            if (compare(result, v) > 0) {
                result = v;
            }
        }
        return result;
    };
}
>> minBy((a, b) => a - b)(10, 20, -30, 40)
-30
>> comparing = (f) => ((a, b) => f(a) - f(b))
>> minByAbs = minBy(comparing(Math.abs))
>> minByAbs(10, 20, -30, 40)
40
>> regularMin = minBy(compare(identity))
>> regularMin(10, 20, -30, 40)
-30
\end{minted}
Ещё пример --- \deff{левая свёртка}. Может создать функцию для суммы, максимума, минимума и любой ассоциативной операции, применяемой на множество аргументов:
\begin{minted}{javascript}
foldLeft = (f, zero) => {
    return (...args) => {
        let result = zero;
        for (const v of args) {
            result = f(result, v);
        }
        return result;
    };
}
>> sum = foldLeft((a, b) => a + b, 0)
>> sum(1, 2, 3)
6
\end{minted}
При помощи функций высшего порядка можно сделать универсальную функцию для нахождения производной:
\begin{minted}{javascript}
diff = dx => f => x =>(f(x + dx) - f(x - dx)) / (dx * 2)
>> dSin = diff(1e-7)(Math.sin)
>> dsin(Math.Pi)
-1
\end{minted}
Ещё пример --- \deff{map}. Позволяет применять функцию на все элементы массива:
\begin{minted}{javascript}
map = (f) => {
    return (...args) => {
        const result = [];
        for (const arg of args) {
            result.push(f(arg));
        }
        return result;
    };
}
>> add10 = map((a) => a + 10)
>> add10(10, 20, 30)
[20, 30, 40]
\end{minted}
Ещё полезная функция --- curry. Позволяет передавать аргументы друг за другом, вместо всех сразу:
\begin{minted}{javascript}
>> curry = f => a => b => f(a, b)
>> add = curry((a, b) => a + b)
>> add10 = add(10)
>> add10(20)
30
\end{minted}

\subsection{Объекты}

\begin{minted}{javascript}
>> let p = Object.create({})
>> p[x] = 10
>> p[y] = 20
>> dumpObject(p)
Object
x = 10 
y = 20
\end{minted}

Можно обращаться p.x  и так далее, есили это корректный Java идентификатор. У объекта могут не определены

\begin{minted}{javascript}
>> println(p[z])
undefined
\end{minted}

\begin{minted}{javascript}
for (let prop in p){

}
\end{minted}

Перебирает свойства и печататет их значения
