\subsection{Неизменяемые объекты}
Неизменяемый объект — это объект, состояние которого не может измениться после его создания. Такие объекты особенно полезны в многопоточных приложениях, так как они безопасны для использования несколькими потоками без необходимости синхронизации.

\begin{itemize}
    \item \textbf{Как создать неизменяемый объект} \par
    \begin{enumerate}
        \item Сделайте все поля \texttt{final} и \texttt{private}.
        \item Не предоставляйте методы для изменения состояния объекта.
        \item Убедитесь, что класс не может быть расширен (объявите его \texttt{final}). Если объект содержит изменяемые поля, убедитесь, что они не могут быть изменены после создания объекта.
    \end{enumerate}
    \begin{minted}[linenos, frame=leftline, tabsize=4, fontsize=\ttfamily\small, framesep=4mm, numbersep=4pt]{java}
public final class ImmutablePerson {
    private final String name;
    private final int age;

    public ImmutablePerson(String name, int age) {
        this.name = name;
        this.age = age;
    }

    public String getName() {
        return name;
    }

    public int getAge() {
        return age;
    }
}
    \end{minted}
    
\item \textbf{Конструкторы (Constructors)} \par
    Конструкторы — это специальные методы, которые используются для инициализации объектов класса. Они вызываются автоматически при создании объекта и имеют то же имя, что и класс.
    \begin{enumerate}
        \item Простой конструктор
        \begin{minted}[linenos, frame=leftline, tabsize=4, fontsize=\ttfamily\small, framesep=4mm, numbersep=4pt]{java}
public class MyClass {
    public MyClass() {
        // Инициализация
    }
}
        \end{minted}
        \item Конструктор с параметрами
        \begin{minted}[linenos, frame=leftline, tabsize=4, fontsize=\ttfamily\small, framesep=4mm, numbersep=4pt]{java}
public class MyClass {
    private int value;

    public MyClass(int value) {
        this.value = value;
    }
}
        \end{minted}
    \end{enumerate}
    Есть возможность сделать несколько конструкторов, принимающих различные типы данных.  
    
    \item \textbf{Методы (Methods)} \par
    Методы — это блоки кода, которые выполняют определенные действия и могут возвращать результат. Они позволяют повторно использовать код и структурировать проет.
    \begin{enumerate}
        \item Объявление метода
        \begin{minted}[linenos, frame=leftline, tabsize=4, fontsize=\ttfamily\small, framesep=4mm, numbersep=4pt]{java}
public class MyClass {
    public void myMethod() {
        // Действия
    }
}
        \end{minted}
        \item Вызов метода
        \begin{minted}[linenos, frame=leftline, tabsize=4, fontsize=\ttfamily\small, framesep=4mm, numbersep=4pt]{java}
MyClass obj = new MyClass();
obj.myMethod();
        \end{minted}
    \end{enumerate}

    \item \textbf{Статические методы (Static Methods)} \par
    Статические методы принадлежат классу, а не экземпляру класса. Они могут быть вызваны без создания объекта. Отличается от обычных методов тем, что статические методы могут обращаться только к статическим полям и методам.

    \begin{minted}[linenos, frame=leftline, tabsize=4, fontsize=\ttfamily\small, framesep=4mm, numbersep=4pt]{java}
public class MyClass {
    public static void myStaticMethod() {
        // Действия
    }
}
    \end{minted}

    \item \textbf{@Override} \par
    В Java есть аннотация \texttt{@Override}, необходимая для указания того, что метод в классе переопределяет метод родительского класса.\par
    Она указывается перед функций, переназначающей другую.
    \begin{minted}[linenos, frame=leftline, tabsize=4, fontsize=\ttfamily\small, framesep=4mm, numbersep=4pt]{java}
class SuperClass {
    void display() {
        System.out.println("Метод суперкласса");
    }
}

class SubClass extends SuperClass {
    @Override
    void display() {
        System.out.println("Переопределенный метод подкласса");
    }
}
    \end{minted}
    
\end{itemize}

\subsection{Изменяемые объекты}
Изменяемые объекты — это объекты, состояние которых можно изменять после их создания. В отличие от неизменяемых объектов, изменяемые объекты позволяют изменять их внутренние данные в течение их жизненного цикла.

\begin{minted}[linenos, frame=leftline, tabsize=4, fontsize=\ttfamily\small, framesep=4mm, numbersep=4pt]{java}
public class MutablePerson {
    private String name;
    private int age;

    public MutablePerson(String name, int age) {
        this.name = name;
        this.age = age;
    }

    public String getName() {
        return name;
    }

    public void setName(String name) {
        this.name = name;
    }

    public int getAge() {
        return age;
    }

    public void setAge(int age) {
        this.age = age;
    }
}
\end{minted}

\begin{itemize}
    \item \textbf{Основные отличия от неизменяемых объектов} \par
    \begin{enumerate}
        \item Состояние изменяемых объектов можно изменять после создания, тогда как состояние неизменяемых объектов остается постоянным.
        \item Неизменяемые объекты безопасны для использования в многопоточных приложениях без дополнительной синхронизации, в то время как изменяемые объекты требуют синхронизации для обеспечения потокобезопасности.
        \item Изменяемые объекты могут быть более эффективными в плане производительности, так как они не требуют создания новых объектов при каждом изменении состояния.
    \end{enumerate}

    \item \textbf{Инкапсуляция (Encapsulation)} \par
    Инкапсуляция — это принцип объектно-ориентированного программирования, который заключается в сокрытии внутреннего состояния объекта и предоставлении доступа к нему только через публичные методы. Это помогает защитить данные от некорректного использования и упрощает управление сложностью программы. \par

    Как реализовать инкапсуляцию:
    \begin{enumerate}
        \item Сделайте поля класса \texttt{private}
        \item Предоставьте публичные методы для доступа к этим полям (геттеры и сеттеры)
    \end{enumerate}

    \begin{minted}[linenos, frame=leftline, tabsize=4, fontsize=\ttfamily\small, framesep=4mm, numbersep=4pt]{java}
public class EncapsulatedPerson {
    private String name;
    private int age;

    // Конструктор
    public EncapsulatedPerson(String name, int age) {
        this.name = name;
        this.age = age;
    }

    // Геттер для имени
    public String getName() {
        return name;
    }

    // Сеттер для имени
    public void setName(String name) {
        this.name = name;
    }

    // Геттер для возраста
    public int getAge() {
        return age;
    }

    // Сеттер для возраста
    public void setAge(int age) {
        this.age = age;
    }
}   
    \end{minted}

\end{itemize}