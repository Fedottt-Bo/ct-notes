\subsubsection{Задача 1.}

$f^1 = w^2 + 2w^3 + 2w^4, f^2 = w^1 + w^2 + 3w^4, f^3 = w^1 + w^3 + w^4$ - 1 формы. $\xi_1 = e_1-e_3 + e_4, \xi_2 =-e_1 + e_2 + e_3 + e_4,\xi_3 = 2e_1 + e_2 + e_3$, $\dim V =4$

Найти
\begin{enumerate}
    \item внешнее произведение $f = f^1 \wedge f^2 \wedge f^3$
    \item значение $f(\xi_1,\xi_2,\xi_3)$
\end{enumerate}

\textbf{Решение:}

Выпишем наши функции в стандартном базисе $f^1 = (0,1,2,2), f^2 = (1,1,0,3), f^3 = (1,0,1,1)$.

Аналогично выпишем $\xi_1= \begin{pmatrix}
    1 \\
    0\\
    -1\\
    1
\end{pmatrix}, \xi_2 = \begin{pmatrix}
    -1 \\
    1 \\
    1 \\
    1\\
\end{pmatrix}, \xi_3 =\begin{pmatrix}
    2\\
    1\\
    0\\
    1\\
\end{pmatrix}$

1) $f=f^1 \wedge f^2 \wedge f^3 = 3! \Alt(f^1 \otimes f^2 \otimes f^3)$. Если вы очень хотите, то можете посчитать. Я таким заниматься не буду. 
$$f = f^1 \wedge f^2 \wedge f^3 = \sum\limits_{j_1<j_2<j_3}\begin{vmatrix}
    a^1_{j_1} & a^1_{j_2} & a^1_{j_3}\\
    a_{j_1}^2 & a^2_{j_2} & a^2_{j_3}\\
     a^3_{j_1} & a^3_{j_2} & a^3_{j_3}\\
\end{vmatrix} w^{j_1}\wedge w^{j_2}\wedge w^{j_3}$$
Такое уже считать гораздо легче, это $\beta = (-3,0,6,-3)  = (\beta_{123}, \beta_{124},\beta_{134},\beta_{234})$. Мы нашли нашу функцию в базисе $p$-форм. В принципе понятно, как ее перевести в базис тензоров.

Также можно было просто в тупую раскрыть: 

$$f^1\wedge f^2 \wedge f^3 = (w^2 + 2w^3 + 2w^4) \wedge  (w^1 + w^2 + 3w^4) \wedge (w^1 + w^3 + w^4)$$
И разложить данную штуку по диструбитивности. Я этим заниматься не буду в экономии своего времени, но как вариант так тоже можно.

2) $f^1 \wedge f^2 \wedge f^3 (\xi_1,\xi_2,\xi_3) = \det \begin{pmatrix}
    f^1(\xi_1) & f^1(\xi_2) & f^1(\xi_3)\\
    f^2(\xi_1) & f^2(\xi_2) & f^2(\xi_3)\\
    f^3(\xi_1) & f^3(\xi_2) & f^3(\xi_3)\\
\end{pmatrix}$


Или можно например использовать формулу:
$$ f^1 \wedge f^2 \wedge f^3 (\xi_1,\xi_2,\xi_3) = \sum\limits_{j_1<j_2<j_3}\begin{vmatrix}
    a^1_{j_1} & a^1_{j_2} & a^1_{j_3}\\
    a_{j_1}^2 & a^2_{j_2} & a^2_{j_3}\\
     a^3_{j_1} & a^3_{j_2} & a^3_{j_3}\\
\end{vmatrix} \cdot \begin{vmatrix}
    \xi_1^{j_1} & \xi_2^{j_1} & \xi_3^{j_1}\\
     \xi_1^{j_2} & \xi_2^{j_2} & \xi_3^{j_2}\\
      \xi_1^{j_3} & \xi_2^{j_3} & \xi_3^{j_3}\\
\end{vmatrix}$$
Ответом будет $-27$.
\subsubsection{Задача 2.}

$f = w^1 +w^2 + 2w^3, g = w^1+3w^2 + w^3, h= w^1 + w^3$ - 1 формы. $\dim V =3 $. Найти внешнее произведение:

\textbf{Решение:}

Конечно, мы можем пользоваться решениями из прошлых пунктов но это крайне скучно, поэтому мы воспользуемся одним примером, что тк у нас 3-форма и $\dim V = 3$, то будет выполнено: 
$$f\wedge g \wedge h = \beta_{123} w^1\wedge w^2 \wedge w^3  = \det \begin{pmatrix}
    a^1_{1} & a^1_{2} & a^1_{3}\\
    a_{1}^2 & a^2_{2} & a^2_{3}\\
     a^3_{1} & a^3_{2} & a^3_{3}\\
\end{pmatrix} = -3$$
\subsubsection{Задача 3.}

$\dim V = 4, f \in \Lambda^2V, f = w^1\wedge w^2 + w^1 \wedge w^3 + w^1\wedge w^4 + w^2\wedge w^3 + w^3 \wedge w^4$. Найти представления в базисах пространства $V$(тензоров) и $p$-форм.

\textbf{Решение:}

Решение здесь будет крайне тривиальным. Найдем сначала представление в базисе $p$-форм. Мы получим, что у нас оно $\beta= (1,1,1,1,0,1)$. И теперь восстановим нашу матрицу в пространстве тензоров:
$$\alpha =\begin{pmatrix}
    0 & 1 & 1 & 1\\
    -1 & 0 & 1 & 0 \\
    -1 & -1 & 0 & 1 \\
    -1 & 0 & -1 & 0
\end{pmatrix}$$

