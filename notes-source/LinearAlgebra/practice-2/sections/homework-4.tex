\subsubsection{Задание 1.}

Тензор $f \in T(2,0)$ задан матрицей своих компонент $F$, а $g$ - 1 форма, заданная строкой:
$$F = \begin{pmatrix}
    0 & -1 & 2\\
    1 & 0 & -2 \\
    -2 & 2 & 0 
\end{pmatrix}, G =\begin{pmatrix}
    6 & 3 & 3
\end{pmatrix} $$
\begin{enumerate}
    \item Найти внешнее произведение $f \wedge g$
    \item выписать матрицу тензора $f \wedge g$ в базисе $T(3,0)$.
    \item представить f в виде внешнего произведения линейных форм
\end{enumerate}

\textbf{Решение:}

1) Заметим, что $f,g$ - $p$-формы

Внешнее произведение $f  \wedge g = \cfrac{3!}{1!\cdot2!}\Alt (f \otimes g)$

Найдем $f \otimes g = \begin{pmatrix}[ccc|ccc|ccc]
    0 &6 &12 &  0 & -3 & 6 & 0 & -3 & 6\\
    6 & 0 & -12 & 3 & 0 &-6 & 3 & 0 & -6\\
    -12 & 12 & 0 & -6 & 6 & 0 & -6 &6 &0
\end{pmatrix}$

Проальтернируем и получим 
$$\Alt(f\otimes g)=-21 \begin{pmatrix}[ccc|ccc|ccc]
    0 & 0 & 0 & 0& 0 & -1 & 0  & 1 &0 \\
    0 & 0 & 1 & 0 & 0 & 0 & -1 &0 & 0\\
    0 & -1 &0 & 1 & 0 & 0 & 0& 0& 0\\
\end{pmatrix}$$
$$f\wedge g = \beta_{123}w^1\wedge w^2 \wedge w^3 = -21w^1\wedge w^2 \wedge w^3
$$
2) Уже написано выше

3) Как это делать нормально я не знаю. Можно решать уравнения, можно еще что-нибудь:
$$f = - w^1 \wedge w^2 +  2 w^1 \wedge w^2 - 2 w^2 \wedge w^3$$
$$f^1 =\beta_1 w^1 + \beta_2 w^2 + \beta_3 w^3; f^2 = \alpha_1 w^1 + \alpha_2 w^2 + \alpha_3w^3$$
$$f^1 \wedge f^2 =(\beta_1 w^1 + \beta_2 w^2 + \beta_3 w^3) \wedge (\alpha_1 w^1 + \alpha_2 w^2 + \alpha_3w^3) $$
Раскрывайте и решайте уравнение с 6 переменными.

Получите $f = (w^1 +w^2 - 4w^3)\wedge (-w^1-2w^2 + 6w^3)$

\subsubsection{Задание 2.}

Найти существенные координаты внешнего произведение трех векторов:
$$f^1=\begin{pmatrix}
    1 & 1&1&1&1
\end{pmatrix}, f^2 = \begin{pmatrix}
    6&8&0&0&0
\end{pmatrix},f^3 = \begin{pmatrix}
    0 & -3 & 3 & 3 & 0
\end{pmatrix}$$

\textbf{Решение:}

Делаем это крайне в тупую по формуле через определитель(как делали в практике в номере один под пунктом 1) и получаем:

$f^1 \wedge f^2 \wedge f^3 = (-12,-12,-18,0,18,18,0,24,24,0)$



\subsubsection{Задание 3.}

Найти значение 2-формы, заданной своими сущ. координатами
$f = \begin{pmatrix}
    -2 & -1 & 0 & 0 & 1 & 2
\end{pmatrix}$ на векторах $x= (-1,1,1,0)^T, y = (-2,1,1,2)^T$.

\textbf{Решение:}

Воспользуемся формулой из следствия теоремы 1.


