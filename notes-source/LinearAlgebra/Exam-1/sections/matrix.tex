
\subsection{Основные понятия}
\subsubsection{Определение.}
\textbf{\underline{def:} Матрица} --- множество некоторых объектов (элементов), записанных в виде таблицы (не обязательно числа).
\[ A = (a_{ij})_{m \times n} =
    \begin{pmatrix}
        a_{11} & \ldots & a_{1n} \\
        \vdots & \ddots & \vdots \\
        a_{m1} & \ldots & a_{mn}
    \end{pmatrix}\]
m --- число строк
n --- число столбцов
"Матрица размерности m на n"

Матрица, где $ \forall i,j\ a_{ij} \in \mathbb{R}(\mathbb{C}) $ --- числовая (вещественная/комплексная).


\( A =
\begin{pmatrix}
    A_{1} & \ldots & A_{m}
\end{pmatrix}\) --- столбцовый вид записи.
$A_j$ --- столбец матрицы.
\( A_j =
\begin{pmatrix}
    a_{1j} \\
    \vdots \\
    a_{mj}
\end{pmatrix}
\in \mathbb{R}^m(\mathbb{C}^m)\)

\( A =
\begin{pmatrix}
    S_1    \\
    \vdots \\
    S_m
\end{pmatrix}\) --- строчный вид записи.
$S_i$ --- строка матрицы.
\( S_i =
\begin{pmatrix}
    a_{i1} & \ldots & a_{in}
\end{pmatrix}
\in \mathbb{R}_n(\mathbb{C}_n)\)

\( span(A_1, \ldots, A_n) \subset \mathbb{R}^m(\mathbb{C}^m)\) --- пространство столбцов матрицы


\( A =
\begin{pmatrix}
    a_{11} & *      & *      \\
    *      & \ddots & *      \\
    *      & *      & a_{mn}
\end{pmatrix}\) --- главная диагональ.


\( A =
\begin{pmatrix}
    *      & *       & a_{1n} \\
    *      & \dots & *      \\
    a_{m1} & *       & *
\end{pmatrix}\) --- побочная диагональ.

\( \forall i \neq j \quad a_{ij} = 0\:
\begin{pmatrix}
    a_{11} & 0      & 0      \\
    0      & \ddots & 0      \\
    0      & 0      & a_{mn}
\end{pmatrix} =
\begin{pmatrix}
    \alpha_{1} & 0      & 0          \\
    0          & \ddots & 0          \\
    0          & 0      & \alpha_{n}
\end{pmatrix}
= diag(\alpha_1, \ldots \alpha_n)
\) --- диагональная матрица.

\( E = diag(\alpha_1, \ldots \alpha_n), \forall i\: \alpha_i=1\) --- единичная матрица.


\(
\begin{pmatrix}
    a_{11} & *      & *      \\
    0      & \ddots & *      \\
    0      & 0      & a_{nn}
\end{pmatrix}
\) --- верхнетреугольная матрица.

\(
\begin{pmatrix}
    a_{11} & 0      & 0      \\
    *      & \ddots & 0      \\
    *      & *      & a_{nn}
\end{pmatrix}
\) --- нижнетреугольная матрица.

\subsubsection{Основные операции с матрицами}
\(a_{ij} \in \mathbb{K} \newline
A_{m \times n}, B_{m \times n} \newline\)
\textbf{\underline{def:}} \( C = A + B = (c_{ij})\quad \forall i,j \  c_{ij}=a_{ij}+b_{ij} \newline\)
'+' --- сложение матриц (одной размерности)
\( \newline 0 \) --- нейтральный элемент относительно сложения

\( \lambda \in \mathbb{K} \newline
C = \lambda A = (\lambda a_{ij})\)
$\newline \lambda \times$ --- умножение на скаляр.
\(\newline -1 A \) --- противоположная A матрица (не путать с обратной)


\textbf{Свойства:}
\begin{enumerate}
    \item $ A+B = B+A $
    \item $ (A+B)+C = A+(B+C) $
    \item $ \exists \ 0 $
    \item $ \exists \ -A $
    \item $ \alpha(A + B) = \alpha A + \alpha B $
    \item $ (\alpha + \beta)A = \alpha A + \beta A $
    \item $ (\alpha \beta)A = \alpha( \beta A ) = \beta( \alpha A )$
    \item $ 1A = A $

\end{enumerate} => Линейное пространство (8 аксиом выполнены) $ M_{m \times n}$

\( E_{ij} =
\begin{pmatrix}
    0 & 0       & 0           & 0       & 0 \\
    0 & \ddots  & \vdots      & \dots & 0 \\
    0 & \ldots  & a_{ij} == 1 & \ldots  & 0 \\
    0 & \dots & \vdots      & \ddots  & 0 \\
    0 & 0       & 0           & 0       & 0
\end{pmatrix}
\) --- канонический базис пространства $ M_{m \times n}$
\( A = \displaystyle{\sum^m_{i=1}\sum^{n}_{j=1}a_{ij}E_{ij}} = (\alpha_{ij})_{m \times n} = 0_{m \times n} \Leftrightarrow a_{ij} = 0 \ \forall i,j\)

\( A =
\begin{pmatrix}
    a_{11} \\
    \vdots \\
    a_{1n} \\
    a_{21} \\
    \vdots \\
    a_{2n} \\
    a_{m1} \\
    \vdots \\
    a_{mn}
\end{pmatrix} \leftrightarrow \mathbb{R(C)}^{mn}
\quad \Rightarrow \quad A \cong \mathbb{K}^{mn}
\quad \Rightarrow \quad dim(M_{m \times n}) = mn\)

\textbf{\underline{def:}} Матрицы A и B согласованы, если число столбцов A совпадает с числом столбцов B.

Если A и B согласованы, то
\( A_{m \times k}, B_{k \times n} \newline
C = A \times B = A B = (C_{ij})_{m \times n} \quad C_{ij} = \displaystyle{\sum^{k}_{r=1}a_{ir}b_{rj}} \)
--- умножение

\textbf{\underline{def:}} Матрицы A и B перестановочны, если $ AB = BA $ (очевидно, должны быть квадратными)

A, B, C --- квадратные матрицы $ n \times n \newline
    \forall \lambda \in \mathbb{K}$
\begin{enumerate}
    \setcounter{enumi}{8}
    \item $ A(B+C) = AB + AC \newline
              (A+B)C = AC + BC $
\end{enumerate} => кольцо

\begin{enumerate}
    \setcounter{enumi}{9}
    \item $ \lambda (AB) = (\lambda A)B = A(\lambda B)$
\end{enumerate} => алгебра ($ M_{n \times n} $)

\begin{enumerate}
    \setcounter{enumi}{10}
    \item $ A(BC) = (AB)C$
\end{enumerate} => ассоциативная алгебра

\begin{enumerate}
    \setcounter{enumi}{12}
    \item $ \exists E \quad EA=AE=A$
\end{enumerate} => унитальная алгебра

(Обратный элемент может не существовать, так что без 12)

\textbf{Доказательства:} упражнение на дом :) Но вообще там несложно, просто глина.
\subsection{Операция транспонирования и её свойства.}
\deff{Операция транспонирования} заменяет матрицу $ A_{m \times n} $ на $ A^T_{n \times m} $, где строки новой матрицы - столбцы исходной (проще говоря, отражение относительно главной диагонали)

\( B = A^T = (b_{ij}) = (a_{ji}) \)

\textbf{Свойства:}
\begin{enumerate}
    \item $ (A^T)^T = A $
    \item $ (A+\lambda B)^T = A^T+\lambda B^T $
    \item A и B согласованы $ (AB)^T = B^T A^T $ (!!! не путать, я так вторую попытку кр по матрицам слил)
\end{enumerate}

\textbf{\underline{def:}} $ A_{m \times n} $ называется симметрической, если $ A = A^T$

\textbf{\underline{def:}} $ A_{m \times n} $ называется кососимметрической, если $ A = -A^T$
\subsection{Обратная матрица и её свойства.}

\( A_{n \times n}\)
Матрица $ A^{-1} $ называется \deff{обратной} к $ A $, а $ A $ называется обратимой, если $ A^{-1}A = AA^{-1} = E $

Пока мы не знаем условий существования (в лекциях позже)

\textbf{Свойства:}
\begin{enumerate}
    \item $ A^{-1} $ --- единственная (док-во очевидное через ассоциативность)
    \item $ (A^{-1})^{-1} = A $ (из определения)
    \item $ \forall \lambda, \lambda \in \mathbb{K} \quad (\lambda A)^{-1} = \frac{1}{\lambda}A^{-1}$
    \item $ E^{-1} = E $
    \item $ (A^T)^{-1} = (A^{-1})^T $
    \item $ \exists B^{-1} \Rightarrow \exists (AB)^{-1} = B^{-1}A^{-1}$
\end{enumerate}

\subsection{Ранг матрицы.}
\textbf{\underline{def:}} $ rg_{line}(A) = rg(S_1, \ldots S_n)$ --- строчный ранг матрицы A (берем строки как вектора, находим ранг системы векторов) $ 1 \leq rg_{line}(A) \leq n \  (A \neq 0)$

\textbf{\underline{def:}} $ rg_{col}(A) = rg(A_1, \ldots A_m)$ --- столбцовый ранг матрицы A (берем строки как вектора, находим ранг системы векторов) $ 1 \leq rg_{col}(A) \leq m \  (A \neq 0)$

\( A_j \quad 1 \leq i_1 < \ldots <i_k \leq m \newline
\widetilde{A}_j =
\begin{pmatrix}
    a_{i_1j} \\
    a_{i_2j} \\
    \vdots   \\
    a_{i_kj}
\end{pmatrix}
\) --- отрезок длины k столбца $ A_j $

\( S_i \quad 1 \leq j_1 < \ldots < j_k \leq n \newline
\widetilde{S}_i =
\begin{pmatrix}
    a_{ij_1} & a_{ij_2} & \ldots & a_{ij_k}
\end{pmatrix}
\) --- отрезок длины k строки $ S_i $



\textbf{\underline{Утверждение 1:}} $ A_1, A_2, \ldots A_n $ линейно зависимы => любые отрезки длины k $ \widetilde{A}_1 \ldots \widetilde{A}_n $ линейно зависимы. \newline Доказательство от противного: предполагаем, что независимы, но у нас уже есть нетривиальная линейная комбинация столбцов $ A_1, A_2, \ldots A_n $, равная нулю, и если мы удалим часть строк, линейная комбинация всё так же будет давать 0. %TODO: если не очевидно - напишите @Blobbot54rus, я перестану лениться и доделаю

\textbf{\underline{Следствие:}} Отрезки длины k $ \widetilde{A}_1 \ldots \widetilde{A}_n $ линейно независимы = > $ A_1, A_2, \ldots A_n $ линейно независимы.

\textbf{\underline{Утверждение 2:}}
\( rg_{line}(A) = k \quad S_{i_1} \ldots S_{i_k}\) --- база строк. Тогда, если $ \widetilde{A}_1 \ldots \widetilde{A}_n $ отрезки, отвечающие $ S_{i_1} \ldots S_{i_k} $, линейно зависимы, то и $ A_1, A_2, \ldots A_n $ линейно зависимы.

\begin{enumerate}
    \item[] \prooff{}
    н.у.о. $i_1,\ldots,i_k = 1,2,\ldots, k$. Значит все оставшиеся - линейно комбинация.
    Значит я любую строчку могу записать, как линейную комбинацию наших строк:

    $s_{k+l}= \sum\limits_{r=1}^k \alpha_{rl}s_r$. $a_{{k+l}_j}=\sum\limits_{r=1}^k \alpha_{rl}a_{r_j}$:

    $ \widetilde{A}_1 \ldots \widetilde{A}_n $ отрезки, отвечающие $ S_{i_1} \ldots S_{i_k} $, линейно зависимы $\Leftrightarrow$ $\exists \beta_j \in K$ не все нули.

    $\sum\limits_{j=1}^n{b_j \widetilde{A}_j}=0$. Докажем, что с этими же $\beta$

    $\sum\limits_{j=1}^n b_j A_j = 0$. Первые k - нули. Докажем, что и оставшиеся нули.

    Посмотим на $k+1$ координату: $\sum\limits_{j=1}^n \beta_j a_{k+1_{j}} = \sum\limits_{j=1}^n \beta_j \sum\limits_{r=1}^k \alpha_{r1}a_{r_{j}} = \sum\limits_{r=1}^k \alpha_{r_1}\sum\limits_{j=1}^n \beta_j a_{r_{j}} =0$

    Далее аналогично

    
\end{enumerate}
\deff{Теорема} (о ранге матрицы)

\( rg_{line}(A) = rg_{col}(A) = rg(A) \)
\begin{enumerate}
    \item[] \prooff{}
    $\rg A = k: 1 \leq k \leq n,m$

    н.у.о. Пускай первые k строк линейно независимы.
    
    Рассмотрим отрезки столбиков, соответсвующие этим элементам

    $r = \rg (\widetilde{A}_1 \ldots \widetilde{A}_n) \leq k$. Тогда докажем, что ранг исходных столбиков не превосходит k. Пускай ранг r, тогда есть база, тогда исходные столбики  тоже лин. независимы. Любой вектор не входящий в него - линейно зависимый по утверждению 2. Тогда база столбиков соответствует базе подотрезков. Откуда rg столбиков меньше либо равен рангу строк. Аналогично в обратную сторону, выиграли.
\end{enumerate}

\subsection{Свойства ранга. Теорема о приведении матрицы к трапецевидной}

\textbf{Свойства ранга:}
\begin{enumerate}
    \item $ rg(A^T) = rg(A) $
    \item $ rg(\lambda A) = rg(A) $
    \item $ rg(A+B) \leq rg(A)+rg(B) $ (лайт версия т. Грассмана)
    \item A и B согласованы, $ rg(AB) \leq \min(rg(A),rg(B)) $
\end{enumerate}

\textbf{Матрица трапециевидной формы (н.у.о. n <= m:)}

\(
\begin{pmatrix}
    a_{11} & *      & *      & * & * \\
    0      & \ddots & *      & * & * \\
    0      & 0      & a_{nn} & * & *
\end{pmatrix}
\) 
Очевидно любую матрицу млжно привести к трапецевидной.


