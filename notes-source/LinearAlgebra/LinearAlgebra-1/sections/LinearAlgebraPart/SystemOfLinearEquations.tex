\section{Системы линейных алгебраических уравнений (СЛАУ)}
\subsection{Основные определения и понятия, теорема Кронекера-Капелли.}
Обычно система записывается так:
\(\begin{cases}
    a_{11}x_1 + a_{12} + \cdots + a_{1n} x_n = b_1 \\

    a_{21}x_1 + a_{22} + \cdots + a_{2n} x_n = b_2 \\

    \vdots                                         \\

    a_{m1}x_1 + a_{m2} + \cdots + a_{mn} x_n = b_m \\
\end{cases}\)

\textbf{Матричная форма записи} --- \(Ax = b\), где

\(A = (a_{ij})_{m\times n} =
\begin{pmatrix}
    a_{11} & \ldots & a_{1n} \\
    \vdots & \ddots & \vdots \\
    a_{m1} & \ldots & a_{mn}
\end{pmatrix}\),
\(x = \begin{pmatrix}
    x_1 \\
    x_2 \\ \vdots \\ x_n \\
\end{pmatrix}\),
\(b = \begin{pmatrix}
    b_1 \\
    b_2 \\ \vdots \\ b_m \\
\end{pmatrix}\)

\(Ax = b\), где \(A = (A_1, \ldots, A_n)\) - столбики --- матричная запись.


\(Ax = b\) --- \textbf{система однородных линейных уравнений (СЛОУ) (однородная система)}, если \(b = \mathbb{0}\).

\(Ax = b\) --- \textbf{система неоднородных линейных уравнений (СЛНУ) (неоднородная система)}, если \(b \neq \mathbb{0}\).

Система \(Ax = b\) --- \textbf{совместная (разрешенная)}, если \(\exists x\), то есть существует решение.

Система \(Ax = b\) --- \textbf{несовместная (неразрешенная)}, если \(\not\exists x\), то есть решения не существует.

\textbf{Замечание}: СЛОУ всегда совместна, т.к. \(x = \mathbb{0}\) всегда является решением.

Система \(Ax = b\) --- \textbf{определенная}, если есть единственное решение.

Система \(Ax = b\) --- \textbf{неопределенная}, если есть более одного решения.

Система \(Ax = \mathbb{0}\) --- \textbf{тривиальная}, если она определённая, то есть единственное решение \(x = \mathbb{0}\).

\textbf{Общее решение системы} \(Ax = b\) --- \(\{\forall x | Ax = b\}\), то есть множество всех его решений.

\textbf{Частное решением системы} \(Ax = b\) --- какое-то конкретное решение \(x\), рассматриваемое в данном контексте.

\textbf{Расширенная матрица системы} --- \((A | b) =
\begin{pmatrix}[ccc|c]
    a_{11} & \ldots & a_{1n} & b_1    \\
    \vdots & \ddots & \vdots & \vdots \\
    a_{m1} & \ldots & a_{mn} & b_m
\end{pmatrix}\)

\textbf{Теорема Кронекера-Капелли}: \(Ax = b\) совместна \(\Leftrightarrow \rg(A) = \rg(A|b)\)

\begin{enumerate}
    \item[] \prooff{}
          \(Ax = b \Leftrightarrow \sum\limits_{i=1}^n x_i A_i = b\) --- линейная комбинация столбцов \(\Leftrightarrow b \in \spann(A_1, \ldots, A_n) \Leftrightarrow \spann(A_1, \ldots, A_n) = \spann(A_1,\ldots A_n, b)\)

          \(\rg(A) = \dim(\spann(A_1,\ldots, A_n)) = \dim(\spann(A_1, \ldots, A_n, b)) = \rg(A | b)\) \(Q.E.D.\)

\end{enumerate}


\subsection{Структура общего решения СЛОУ и СЛНУ. ФСР. Альтернатива Фредгольма.}

\textbf{Теорема}: \(Ax = \mathbb{0}\), \(u,v \in K^n\) --- решения СЛОУ \(\Rightarrow \forall \lambda \in K: \lambda u + v\) --- тоже решение СЛОУ.

\(u, v\) --- решения \(\Rightarrow Au = \mathbb{0}, Av = \mathbb{0}\)

\(A(\lambda u + v) = \lambda Au + Av = \lambda \mathbb{0} + \mathbb{0} = \mathbb{0} \Rightarrow \lambda u + v\) --- тоже решение СЛОУ \(Q.E.D.\)

\textbf{Следствие}: общее решение СЛОУ --- линейное подпространство \(L \subseteq K^n\)

Смотри критерии линейного подпространства.

\textbf{Теорема (размерность общего  решения СЛОУ)}: \(Ax = \mathbb{0}\), \(\rg(A) = k\), \(L\) --- общее решение СЛОУ \(\Rightarrow \dim(L) = n - k = n - \rg(A)\), где \(n\) --- число неизвестных.

\begin{itemize}
    \item \(k = 0\):

          \(A = \mathbb{0}\) \(\forall x \in K^n: Ax = \mathbb{0} \Rightarrow \dim(L) = \dim(K^n) = n - 0 = n - k\)

    \item \(1\leq k < n\):

          Тогда \(\rg(A) = k = \rg_{col}(A) = \rg(A_1, \ldots, A_n)\) --- база столбцов из \(k\) элементов. Не умаляя общности переставим столбцы чтобы базисом были столбцы \(A_1,\ldots, A_k\), а все остальные столбцы будут их линейными комбинациями.

          \(A_{k + j} = \sum\limits_{i = 1}^k \alpha_i^j A_j\), где \(\alpha_i^j \in K\). (\(j\) --- тоже индекс, просто для удобства записанный сверху)

          \(\sum\limits_{i = 1}^k \alpha_i^j A_i - A_{k + j} = \mathbb{0} \Leftrightarrow
          u_1 = \begin{pmatrix}
              \alpha_1^1 \\
              \alpha_2^1 \\
              \vdots     \\
              \alpha_k^1 \\
              -1         \\
              0          \\
              \vdots     \\
              0
          \end{pmatrix},
          u_2 = \begin{pmatrix}
              \alpha_1^2 \\
              \alpha_2^2 \\
              \vdots     \\
              \alpha_k^2 \\
              0          \\
              -1         \\
              \vdots     \\
              0
          \end{pmatrix}, \ldots,
          u_{n - k} = \begin{pmatrix}
              \alpha_1^{n - k} \\
              \alpha_2^{n - k} \\
              \vdots           \\
              \alpha_k^{n - k} \\
              0                \\
              0                \\
              \vdots           \\
              -1
          \end{pmatrix}\)

          \(u_1, \ldots, u_{n - k}\) --- решения \(Ax = \mathbb{0}\), причём линейно независимые из-за нулевых координат в нижней части векторов.

          Покажем, что \(u_1, \ldots, u_{n - k}\) --- порождающая система. Пусть \(u\) --- решение \(Ax = \mathbb{0}\).

          \(u =
          \begin{pmatrix}
              \beta_1       \\
              \beta_2       \\
              \vdots        \\
              \beta_k       \\
              \beta_{k + 1} \\
              \vdots        \\
              \beta_n
          \end{pmatrix} \Rightarrow v = u + \sum\limits_{i = 1}^{n - k} \beta_{k + j} u_j =\)

          \(=
          \begin{pmatrix}
              \beta_1       \\
              \beta_2       \\
              \vdots        \\
              \beta_k       \\
              \beta_{k + 1} \\
              \beta_{k + 2} \\
              \vdots        \\
              \beta_n
          \end{pmatrix} +
          \begin{pmatrix}
              \beta_{k + 1} \alpha_1^1 \\
              \beta_{k + 1} \alpha_2^1 \\
              \vdots                   \\
              \beta_{k + 1} \alpha_k^1 \\
              -\beta_{k + 1}           \\
              0                        \\
              \vdots                   \\
              0
          \end{pmatrix} +
          \begin{pmatrix}
              \beta_{k + 2} \alpha_1^2 \\
              \beta_{k + 2} \alpha_2^2 \\
              \vdots                   \\
              \beta_{k + 2} \alpha_k^2 \\
              0                        \\
              -\beta_{k + 2}           \\
              \vdots                   \\
              0
          \end{pmatrix} + \cdots +
          \begin{pmatrix}
              \beta_n \alpha_1^{n - k} \\
              \beta_n \alpha_2^{n - k} \\
              \vdots                   \\
              \beta_n \alpha_k^{n - k} \\
              0                        \\
              0                        \\
              \vdots                   \\
              -\beta_n
          \end{pmatrix} =
          \begin{pmatrix}
              \gamma_1 \\
              \gamma_2 \\
              \vdots   \\
              \gamma_k \\
              0        \\
              0        \\
              \vdots   \\
              0
          \end{pmatrix}\)

          \(v\) --- тоже решение \(Ax = \mathbb{0}\), так как является суммой других решений \(Ax = \mathbb{0}\), домноженных на некоторые коэффициенты.

          \(Av = \gamma_1 A_1 + \cdots + \gamma_k A_k = \mathbb{0}\) --- нулевая линейная комбинация линейно независимых векторов \(\Rightarrow \forall \gamma_j = 0 \Rightarrow u + \sum\limits_{i = 1}^{n - k} \beta_{k + j} u_j = \mathbb{0} \Rightarrow u = \sum\limits_{i = 1}^{n - k} (-\beta_{k + j}) u_j \Rightarrow\)

          \(\Rightarrow u_1, \ldots, u_{n - k}\) --- порождающая система \(\Rightarrow u_1, \ldots, u_{n - k}\) --- базис \(L \Rightarrow\)

          \(\Rightarrow \dim L = n - k\)
    \item \(k = n\):

          \(A_1,\ldots, A_n\) --- линейно независимы

          \(Ax = 0 \Leftrightarrow \sum\limits_{i = 1}^n x_i A_i = 0 \Leftrightarrow \forall i = 1, \ldots, n: x_i = 0 \Leftrightarrow x = \mathbb{0}\) --- единственное решение \(\Leftrightarrow \dim L = 0\)
\end{itemize}

\textbf{Следствие}: \(Ax = 0\), \(n\) --- число переменных.

\begin{itemize}
    \item \(0 \leq \rg(A) < n \Rightarrow\) система неопределенная, имеет бесконечно много решений, образующие линейное подпространство.

    \item \(\rg(A) = n \Rightarrow\) система определенная, имеет единственный корень равный нулю, то есть система тривиальная.
\end{itemize}

\textbf{Фундаментальная система решения} --- базис линейного подпространства решений СЛОУ.

\textbf{Теорема (о структуре решения СЛНУ)}: Пусть Ax = b совместна, \(x_0\) --- частное решение СЛНУ: \(x\) --- решение СЛНУ \(\Leftrightarrow x = x_0 + u\), где \(u\) --- некоторое решение \(Ax = \mathbb{0}\)
\begin{itemize}
    \item \(\Rightarrow\):

          \(Ax = b\), \(Ax_0 = b \Rightarrow A(x - x_0) = \mathbb{0} \Rightarrow u = x - x_0\) --- решение \(Ax = \mathbb{0}\)

    \item \(\Leftarrow\):

          \(x = x_0 + u\), \(Au = \mathbb{0}\), \(Ax_0 = b \Rightarrow Ax = A(x_0 + u) = b + \mathbb{0} = b \Rightarrow x\) --- решение \(Ax = b\)
\end{itemize}
\textbf{Следствия}:
\begin{enumerate}
    \item Общее решение \(Ax = b\) --- линейное многообразие \(P = L + x_0\), где \(x_0\) --- частное решение СЛНУ, \(L = \spann(u_1, \ldots, u_{n-k})\) --- общее решение \(Ax = \mathbb{0}\)

          \(\dim(P) = \dim(L)\) --- размерность общего решения СЛНУ.

    \item
          \begin{itemize}
              \item \(0 \leq \rg(A) < n \Rightarrow Ax = b\) имеет бесконечно много решений, \(\dim(P) = n - \rg(A)\)

              \item \(\rg(A) = n \Rightarrow Ax = b\) имеет единственное решение, \(\dim(P) = 0\)
          \end{itemize}
\end{enumerate}

\textbf{Теорема (Альтернатива Фредгольма)}: Пусть \(A_{m\times n} \neq \mathbb{0}\), \(x \in K^n\), \(y \in K^m\): Либо \(\forall b \in K^m: Ax = b\) имеет решение, либо \(A^Ty = \mathbb{0}\) нетривиальна.

То есть, \(\forall b \in K^m\), существует решение \(Ax = b \Leftrightarrow A^Ty = \mathbb{0}\) тривиальна.

\begin{itemize}
    \item \(\Rightarrow\)

          \(\forall b \in K^m Ax = b\) совместно \(\Leftrightarrow b = \sum\limits_{i=1}^n x_iA_i  \Rightarrow b \in span(A_1,\ldots,A_n)\)

          Пусть \(b = E_j = \begin{pmatrix}
              0 \\ \vdots\\a_j\\ \vdots\\ 0 \\
          \end{pmatrix}\), где \(a_j = 1\) - элемент j-ой строки

          \(E_j \in \spann(A_1,\ldots, A_n)\)

          Заметим, что \(K^m \subset span(A_1,\ldots A_m) \subset K^m\), потому что любой базисный вектор  содержится в нашей оболочке. Откуда:

          \(span(A_1,\ldots A_n)  = K^m \Rightarrow rg A = m = rg A^T \Rightarrow A^Ty=0\) будем иметь одно решение, по ранее сказанной теореме.

    \item \(\Leftarrow:\)
          Заметим, что все переходы сверху работают в обе стороны.

\end{itemize}

\subsection{Метод Гаусса решения СЛНУ}

\(Ax=b\).

\textbf{Элементарным преобразованием системы} будем называть:

\begin{enumerate}
    \item добавление / удаление уравнения с нулевыми коэффициентами и нулевым свободным членом.
    \item изменение нумераций уравнений.
    \item умножение \(\forall\) уравнения на \(\forall \lambda \in K, \lambda \neq 0\).
    \item замена \(\forall\) уравнения на его сумму с другим уравнением.
    \item изменение нумерации переменных.
\end{enumerate}

\textbf{Замечание:}
\begin{enumerate}
    \item все элементарные преобразования приводят к эквивалентной системе.
    \item все элементарные преобразования эквиваленты элементарным преобразованиям \(A | b\) и перестановкой в ней столбцов (пункт 5).
\end{enumerate}

\textbf{Теорема (прямой ход метода Гаусса)}

\(\forall Ax = b\)

Элементарными преобразованиями системы исходная система может быть замена на эквивалентную систему, матрица которой будет иметь трапециевидную форму.

\begin{itemize}
    \item Находим в необработанной части матрицы самую левую верхнюю ненулевую ячейку. Переставляем её в самый левый верхний угол необработанной части матрицы.

    \item Отнимаем от всех строчек, ниже первой необработанной, первую необработанную, домноженную на нужный коэффициент, чтобы первый столбец необработанной части оказался заполненным нулями, кроме первой ячейки.

    \item Отмечаем верхнюю необработанную строчку и левый необработанный столбец, как обработанные.
\end{itemize}

\textbf{Метод Гаусса решения СЛАУ}:

\begin{enumerate}
    \item Прямой ход

          См. теорему о приведении матрицы к трапециевидной форме. Проводить её мы будем с расширенной матрицей системы. Один лишь нюанс в том, что переставлять столбец \(B\) ни с чем нельзя, то есть на нём мы заканчиваем алгоритм.

    \item Обратный ход

          \begin{enumerate}
              \item Вид матрицы треугольный

                    Обнулим последний столбец при помощи последней строки:

                    \(
                    \begin{pmatrix}[cccc|c]
                        a_{11} & a_{12} & \cdots & a_{1n} & b_1    \\
                        0      & a_{22} & \cdots & a_{2n} & b_2    \\
                        \vdots & \vdots & \ddots & \vdots & \vdots \\
                        0      & 0      & \cdots & a_{nn} & b_n
                    \end{pmatrix} \sim
                    \begin{pmatrix}[cccc|c]
                        a_{11} & a_{12} & \cdots & 0      & b_1 - b_n \cfrac{a_{2n}}{a_{nn}} \\
                        0      & a_{22} & \cdots & 0      & b_2 - b_n \cfrac{a_{2n}}{a_{nn}} \\
                        \vdots & \vdots & \ddots & \vdots & \vdots                           \\
                        0      & 0      & \cdots & a_{nn} & b_n
                    \end{pmatrix}
                    \)

                    Повторим для предпоследней строки и столбца и так далее. В конце концов придём к виду:

                    \(
                    \begin{pmatrix}[cccc|c]
                        1      & 0      & \cdots & 0      & b_1'   \\
                        0      & 1      & \cdots & 0      & b_2'   \\
                        \vdots & \vdots & \ddots & \vdots & \vdots \\
                        0      & 0      & \cdots & 1      & b_n'
                    \end{pmatrix}
                    \)

                    Значит \(
                    \begin{pmatrix}
                        b_1'   \\
                        b_2'   \\
                        \vdots \\
                        b_n'
                    \end{pmatrix}
                    \) --- решение СЛАУ.


              \item Вид матрицы не треугольный

                    Возьму из матрицы треугольник, а остальные переменные временно занулим. Так найдем одно решение.
          \end{enumerate}
\end{enumerate}

\subsection{Нахождение обратной матрицы методом Гаусса.}

|\(A_{n \times n}\)|. Найти \(A^{-1}_{n\times n}\), такую, что \(A \times A^{-1} = E\)

\(A^{-1}\) - \(n\) неизвестных столбцов. \(A^{-1} =(X_1,\ldots, X_n) = X\)
\(
\begin{pmatrix}
    x_{11} & x_{12} & \cdots & x_{1n} \\
    x_{21} & x_{22} & \cdots & x_{2n} \\
    \vdots & \vdots & \ddots & \vdots \\
    x_{n1} & x_{n2} & \cdots & x_{nn}
\end{pmatrix}
\)

Заметим, что \(A^{-1}\) - решение уравнения
\(AX = E \Leftrightarrow
\begin{cases}
    AX_1 = E_1 \\
    AX_2 = E_2 \\
    \vdots     \\
    AX_n = E_n
\end{cases}
\)

В процессе нахождения неизвестных столбцов мы делаем с левой частью матрицы одни и те же преобразования. Давайте решать \(n\) систем одновременно:

\(
\begin{pmatrix}[cccc|cccc]
    a_{11} & a_{12} & \cdots & a_{1n} & 1      & 0      & \cdots & 0      \\
    a_{21} & a_{22} & \cdots & a_{2n} & 0      & 1      & \cdots & 0      \\
    \vdots & \vdots & \ddots & \vdots & \vdots & \vdots & \ddots & \vdots \\
    a_{n1} & a_{n2} & \cdots & a_{nn} & 0      & 0      & \cdots & 1
\end{pmatrix} \sim
\begin{pmatrix}[cccc|cccc]
    1      & 0      & \cdots & 0      & x_{11} & x_{12} & \cdots & x_{1n} \\
    0      & 1      & \cdots & 0      & x_{21} & x_{22} & \cdots & x_{2n} \\
    \vdots & \vdots & \ddots & \vdots & \vdots & \vdots & \ddots & \vdots \\
    0      & 0      & \cdots & 1      & x_{n1} & x_{n2} & \cdots & x_{nn}
\end{pmatrix}
\)

\textbf{Теорема. (о существовании обратной матрицы)}

Дано: матрица \(A_{n \times n}\)

\(\exists A^{-1}\) (A обратима) \(\Leftrightarrow rg A = n\)

Причем \(A^{-1}\) может быть найден методом Гаусса.
\begin{enumerate}
    \item[] \prooff{}
          Такая $A^{-1}$ если есть решения $AX_i = E_i$, это значит, что $\rg(A|E_i) = \rg A$, откуда каждый $E_i$ в спане. Откуда, $rg A = n$.
\end{enumerate}

\textbf{Следствие}. Дано \(A_{n \times n}, Ax=b\). \(A\) обратимо \(\Leftrightarrow\) существует единственное решение СЛНУ. Причем, \(x = A^{-1} b\)

\(A\) обратима \(\Leftrightarrow\) \(rg A = n \Leftrightarrow\) существует единственное решение СЛНУ \(\Leftrightarrow A^{-1}\).

\(Ax = b \Leftrightarrow A (A^{-1}b) = b \Leftrightarrow b = b\) Q.E.D

\textbf{Теорема (о ранге произведения матрицы и обратимой матрицы)}

\(A_{n\times n}\), A - обратима, \(B_{m \times n} \Rightarrow
\begin{cases}
    \rg(AB) = \rg B \\
    \rg(BA) = \rg B
\end{cases}
\)
\begin{enumerate}
    \item[] \prooff{}
          $\rg(AB)\leq (\rg A, \rg B)\leq \rg B$.

          $\rg B = \rg EB = \rg (A^{-1}AB)\leq rg(AB) \leq \rg B$
\end{enumerate}


\subsection{Геометрическая интерпретация СЛАУ}

\(V, \dim V = n\)

\(e_1,e_2,\ldots, e_n\) - базис

Множество точек пр-ва V, координаты которых удовлетворяют алгебраическому уравнению 1-ой степени( линейному) наз-ся \uline{\textbf{гиперплоскостью}} в пр-ве V.

$\forall x \in V \leftrightarrow x \in \mathbb{R}^n(\mathbb{C}^n)$ --- координатный изоморфизм

$\alpha_1,\ldots,\alpha_m$ - гиперплоскости

Что будет в пересечении $\alpha_1 \cap \alpha_2 \cap \ldots \cap \alpha_m$ ?

%TODO: матрица a11x1 + a1xn = b1  и так далее

%Тут Кучерук приводит пример при n=3

\(
\begin{pmatrix}[ccc|c]
    a_{11}^{\prime} & a_{12}^{\prime} & a_{13}^{\prime} & b_{1}^{\prime} \\
    a_{21}^{\prime} & a_{22}^{\prime} & a_{23}^{\prime} & b_{2}^{\prime} \\
    0               & 0               & 0               & b_{3}^{\prime} \\
    \vdots          & \vdots          & \vdots          & \vdots         \\
    0               & 0               & 0               & b_{k}^{\prime} \\
\end{pmatrix}
\)

\begin{enumerate}
    \item
    \item
    \item \(\rg A = 3 = \rg A | B\), это значит, что есть 3 линейно независимые строки, а остальные - их лин. комбинация.

          То есть существуют 3 некомпланарные нормали \(\vec N_1, \vec N_2, \vec N_3\).

          Прямые лежащие в попарном пересечении плоскостей с этими нормалями будут не параллельными, то есть т.к. система совместна, то существует точка, принадлежащая каждой из прямых, т.е. все 3 прямые пересекаются в 1-ой точке.
\end{enumerate}

\subsection{Матрица перехода от старого базиса к новому. Связь координат вектора в разных базисах.}

\(V, e_1, e_2, \ldots , e_n\) - старый базис - \(E\).

\(e_1',e_2',\ldots e_n'\) - новый базис - \(E'\).

\(x\in V \leftrightarrow x = \begin{pmatrix}
    x_1    \\
    x_2    \\
    \vdots \\
    x_n
\end{pmatrix} \in K^n\)  - координаты в базисе \(E\).

\(x\in V \leftrightarrow x = \begin{pmatrix}
    x_1'   \\
    x_2'   \\
    \vdots \\
    x_n'
\end{pmatrix} \in K^n\)  - координаты в базисе \(E'\).

\(x = \sum\limits_{i=1}^n x_i e_i = \sum\limits_{i=1}^n x_i' e_i'\).

Давайте представим \(e_j'\) через старый базис: \(T_j = \begin{pmatrix}
    t_{1j} \\
    \vdots \\
    t_{nj}
\end{pmatrix}\) - координаты в базисе \(e\).

\(T = T_{e \rightarrow e'} = (T_1, T_2, \ldots, T_n)\)

\((e_1', \ldots, e_n') = (e_1, \ldots, e_n) T_{e \rightarrow e'}\)

\textbf{Свойства T:}

\begin{enumerate}
    \item \(\rg T = n\) (T обратима)


    \item \(T^{-1}\) - матрица перехода из \(e_1'\) в\(e_1\).

          Пусть B - матрица перехода от e' к e.

          \((e_1,\ldots,e_n) = (e_1',\ldots,e_n')B = ((e_1,\ldots,e_n)T)B = (e_1,\ldots,e_n) (BT)\), откуда \(BT = 1\),  откуда  \(B = T^{-1}\)


    \item связь координат вектора в разных базисах:

          %обозначать координаты X - координатная матрица
          \(x \leftrightarrow X\) в старом базисе

          \(x \leftrightarrow X'\) в новом базисе

          \(x = \sum\limits_{i=1}^n x_i e_i  = \sum \limits_{j=1}^n = x_j'e_j' = \sum\limits_{j=1}^n x_j' \sum\limits_{i=1}^n t_{ij} e_j = \sum\limits_{i=1}^n (\sum\limits_{j=1}^n t_{ij} e_i)\)

          т.е координаты определяются единственный образом

          \(\forall i = 1\ldots n: x_i =\sum\limits_{j=1}^n t_{ij} x_j' = (TX')_i\)

          \(X' = T^{-1} X\)
\end{enumerate}

