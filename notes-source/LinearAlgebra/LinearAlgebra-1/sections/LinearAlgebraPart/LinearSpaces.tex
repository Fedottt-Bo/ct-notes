\section{Линейные пространства.}
\subsection{Основные определения.}
В этом разделе мы будем рассматривать линейные пространства над $\mathbb C$ и иногда $\mathbb R$.  Обозначать над чем мы будем $K$.
\subsubsection{Линейная оболочка, линейная независимость векторов.}
Говорят, что вектор $u$ является \deff{линейной комбинацией} векторов $(v_1;v_2;\ldots;v_n)$, если  $\exists\lambda_1;\lambda_2;\ldots;\lambda_n\in K:$ 
$$ u=\sum\limits_{i=1}^n\lambda_i\cdot v_i$$
Если все \(\lambda_k = 0\), то линейная комбинация называется \deff{тривиальной}

Система векторов \(v_1, \ldots, v_m \in V\) называется \deff{линейной независимой}, если любая нулевая линейная комбинация тривиальна \(\defLeftrightarrow \sum\limits_{k = 1}^{m}\lambda_k v_k = 0 \Leftrightarrow \forall k \in \{1, \ldots, m\}: \lambda_k = 0\)

В противном случае, система векторов называется \deff{линейно зависимой}, т.е. \(\exists\) набор \(\lambda_1, \ldots, \lambda_m\) не все нули таких, что \(\sum\limits_{k = 1}^{m} \lambda_k v_k = 0\).

\(\spann (v_1,v_2,\ldots,v_n) \) это \deff{линейная оболочка} векторов --- множество всевозможных векторов, представимых через $v_1,v_2,\ldots,v_n$.

\subsubsection{Теорема о линейно независимых системах векторов}
\thmm{Теорема}

\begin{enumerate}
    \item \(v_1, \ldots, v_m\) -линейно зависима \(\Leftrightarrow\) по крайней мере один из векторов --- это линейная комбинация остальных

    \item Если некоторая подсистема системы векторов \(v_1, \ldots, v_m\) - линейно зависима, то система векторов \(v_1, \ldots, v_m\) --- линейно зависима
    \item
          \(\begin{rcases*}
              v_1, \ldots, v_m - \text{линейно независима} \\
              v_1, \ldots, v_{m + 1} - \text{линейно зависима}
          \end{rcases*} \Rightarrow\) \(v_{m + 1}\) --- линейная комбинация \(v_1, \ldots, v_m\)
\end{enumerate}

\textbf{Доказательство}

\begin{enumerate}
    \item
          \fbox{\(\Rightarrow\)} \(v_1, \ldots, v_m\) --- линейно зависима, т.е. \(\exists\) нетривиальный набор \(\lambda_1, \ldots, \lambda_m\) такой, что \(\sum\limits_{k = 1}^{m} \lambda_k v_k = 0\)

          н.у.о. пусть \(\lambda_m \neq 0\), тогда \(\lambda_m v_m = -\sum\limits_{k = 1}^{m - 1} \lambda_k v_k\)

          \(v_m = \sum\limits_{k = 1}^{m - 1} \left(- \dfrac{\lambda_k}{\lambda_m} \right) v_k = \sum\limits_{k = 1}^{m - 1} \lambda_k' v_k \defLeftrightarrow v_m \) --- линейная комбинация \(v_1, \ldots, v_m\)

          \fbox{\(\Leftarrow\)} н.у.о. пусть \(v_m = \sum\limits_{k = 1}^{m - 1} \lambda_k v_k\), тогда \(\sum\limits_{k = 1}^{m - 1} \lambda_k v_k - v_m = 0\)

          \(\exists \lambda_1, \ldots, \lambda_{m - 1}, \lambda_m \neq 0\) такой, что \(\sum\limits_{k = 1}^{m} \lambda_k v_k = 0 \defLeftrightarrow v_1, \ldots, v_m \) --- линейно зависима \hfill Q.E.D.


    \item

          н.у.о. пусть \(v_1, \ldots, v_{m'}\) --- линейно зависима \(m' < m\), тогда

          \(\exists\) нетривиальный набор \(\lambda_1, \ldots, \lambda_{m'}: \sum\limits_{k = 1}^{m'} \lambda_k v_k = 0\)

          При \(\lambda_{m' + 1} = 0, \ldots, \lambda_m = 0:\) набор \(\lambda_1, \ldots, \lambda_m\) --- нетривиален

          \(\Rightarrow  \sum\limits_{k = 1}^{m} \lambda_k v_k = 0 \defLeftrightarrow v_1, \ldots, v_m\) --- линейно зависима \hfill Q.E.D.

    \item

          \(v_1, \ldots, v_m, v_{m + 1}\) --- линейно зависима \(\Rightarrow \exists\) нетривиальный набор \(\lambda_1, \ldots, \lambda_{m + 1}: \sum\limits_{k = 1}^{m} \lambda_k v_k + \lambda_{m+1} v_{m+1} = 0\)

          Если \(\lambda_{m + 1} = 0\), тогда набор \(\lambda_1, \ldots, \lambda_m\) --- нетривиален и \(\sum\limits_{k = 1}^{m} \lambda_k v_k = 0 \defLeftrightarrow v_1, \ldots, v_m \) --- линейно зависима. Противоречие.

          Иначе \(v_{m+1} = \sum\limits_{k = 1}^{m} \left( - \dfrac{\lambda_k}{\lambda_{m + 1}} \right) v_k = \sum\limits_{k = 1}^{m} \lambda_k' v_k \defLeftrightarrow v_{m + 1}\) --- линейная комбинация \(v_1, \ldots, v_m\) \hfill Q.E.D.
\end{enumerate}

\textbf{Следствия:}

\begin{enumerate}
    \item Если система линейно независима, то любая подсистема линейно независима.

    \item Если система содержит \(0\) вектор, либо пару пропорциональных векторов, то система линейно зависима.
\end{enumerate}



\subsubsection{Теорема о «прополке».}

Любую систему векторов \(v_1, \ldots, v_m\), в которой хотя бы один из векторов ненулевой, можно заменить на линейно независимую систему векторов \(v_{j_1}, \ldots, v_{j_k}\) с сохранением линейной оболочки. \(\spann (v_1, \ldots, v_m) = \spann (v_{j_1}, \ldots, v_{j_k})\)

\begin{enumerate}
    \item[] \prooff{}
          Пусть \(s_0 = 0, s_1 = \spann (v_1), \ldots, s_m = \spann (v_1, \ldots, v_m)\)

          Тогда \(s_0 \subset s_1 \subset \ldots \subset s_m \subset V\).

          Идём от \(j = m\) до \(j = 2\).

          Если \(s_{j - 1} = s_j\), то \(v_j\) удаляем. При этом \(\spann (v_1, \ldots, v_j) = \spann (v_1, \ldots, v_{j - 1})\) сохраняется.

          Если \(s_{j - 1} \subset s_j\), то \(v_j \notin s_{j - 1}\), т.е. \(v_j\) --- не является линейной комбинацией \(v_1, \ldots, v_{j - 1}\).

          Продолжая так делать, получим, что никакой вектор из полученных не является линейной комбинацией других, то есть итоговое подмножество линейно независимо. В результате получается цепочка строго вложенных подмножеств \(s_0 \subset s_{j_1} \subset \ldots \subset s_{j_k} \subset s_m \subset V \)

          \(\Rightarrow s_m = \spann (v_{j_1}, \ldots, v_{j_k})\) \hfill Q.E.D.
\end{enumerate}

\subsection{Порождающая (полная) система векторов. Базис и размерность линейного пространства}

Система векторов \(v_1, \ldots, v_m \in V\) называется порождающей (полной), если любой вектор линейного пространства \(V\) раскладывается по этим векторам, т.е. является линейной комбинацией \(v_1, \ldots, v_m\). \(V = \spann (v_1, \ldots, v_m)\)

Если число \(v_1, \ldots, v_m\) конечно, то линейное пространство называется конечномерным.

\textbf{Теорема}

Следующие утверждения равносильны:

\begin{enumerate}
    \item \(v_1, \ldots, v_n \in V\) --- линейно независимая и порождающаяся система

    \item \(v_1, \ldots, v_n \in V\) --- линейно независимая система и максимальная по числу элементов

    \item \(v_1, \ldots, v_n \in V\) --- порождающая система и минимальная по числу элементов
\end{enumerate}

\textbf{Доказательство}

\fbox{\(1 \Rightarrow 2\)}
\(v_1, \ldots, v_n\) --- линейно независимая и порождающая система

Пусть \(u_1, \ldots, u_m\) --- линейно независима

Тогда \(\forall u \in V: v_1, \ldots v_n, u\) --- линейно зависима, т.к. \(v_1, \ldots, v_n\) --- порождающая система, то \(u\) --- линейная комбинация \(v_1, \ldots, v_n\), или \(\spann (v_1, \ldots, v_n, u) = V\)

\[
    \underbracket{u_m, v_1, \ldots, v_n}_{\substack{\text{линейно зависима} \\ n + 1}} \xrightarrow{\text{прополка}} \underbracket{u_m, v_1, \ldots}_{\substack{\text{линейно независима} \\ \spann (u_m, v_1, \ldots) = V \\ \le n}}
\]
\[
    \underbracket{u_{m - 1}, u_m, v_1, \ldots}_{\substack{\text{линейно зависима} \\  \le n}} \xrightarrow{\text{прополка}} \underbracket{u_{m - 1}, u_m, v_1, \ldots}_{\substack{\text{линейно независима} \\ \spann (u_{m - 1}, u_m, v_1, \ldots) = V \\ \le n}}
\]
\[\text{и т.д.}\]
\[
    \underbracket{u_1, \ldots, u_m, v_1, \ldots}_{\le n} - \text{линейно независима} \Rightarrow m \le n
\]


\fbox{\(2 \Rightarrow 1\)}
\(v_1, \ldots, v_m\) --- линейно независимая система и максимальная по числу элементов

\(
\forall v \in V: \begin{rcases*}
    v_1, \ldots, v_n - \text{линейно независима} \\
    v_1, \ldots, v_n, v - \text{линейно зависима}
\end{rcases*} \Rightarrow v = \sum_{k = 1}^{n} \lambda_k v_k \Rightarrow v_1, \ldots, v_n - \text{порождающая}
\)

\fbox{\(1 \Rightarrow 3\)}
\(v_1, \ldots, v_n \in V\) --- линейно независимая и порождающаяся система

Пусть \(u_1, \ldots, u_m\) -- порождающая система

\(\spann (u_1, \ldots, u_m) = V\)

\(\forall v \in V: u_1, \ldots, u_m, v\) --- линейно зависима

\[
    \underbracket{v_n, u_1, \ldots, u_m}_{\substack{\text{линейно зависима} \\ m + 1}} \xrightarrow{\text{прополка}} \underbracket{v_n, u_1, \ldots}_{\substack{\text{линейно независима} \\ \spann (v_n, u_1, \ldots) = V \\ \le m}}
\]
\[
    \underbracket{v_{n - 1}, v_n, u_1, \ldots}_{\substack{\text{линейно зависима} \\  \le m + 1}} \xrightarrow{\text{прополка}} \underbracket{v_{n - 1}, v_n, u_1, \ldots}_{\substack{\text{линейно независима} \\ \spann (v_{n - 1}, v_n, u_1, \ldots) = V \\ \le m}}
\]
\[\text{и т.д.}\]
\[
    \underbracket{v_1, \ldots, v_n, u, \ldots}_{n \le m} - \text{линейно независима} \Rightarrow n \le m
\]

\fbox{\(3 \Rightarrow 1\)}

\(v_1, \ldots, v_n\) --- порождающая система и минимальная по числу элементов

Пусть \(v_1, \ldots, v_n\) --- линейно зависима

Тогда \(\exists v\) --- линейная комбинация остальных \(\Rightarrow\) можно сделать прополку

\(v_1, \ldots, v_n \xrightarrow{\text{прополка}} v_{j_1}, \ldots, v_{j_k}\) --- порождающую систему с меньшим числом элементов (при прополке хотя бы один вектор уйдёт)

Но это противоречит тому, что \(v_1, \ldots, v_n\) --- минимальная по числу элементов \(\Rightarrow\) \(v_1, \ldots, v_n\) --- линейно независимая

\hfill Q.E.D.

Если система \(v_1, \ldots, v_n \in V\) удовлетворяет условиям теоремы, то она называется \deff{базисом} пространства \(V\).

Количество векторов \(n = \dim V = \) \deff{размерность линейного пространства} \(= \max\) возможное число линейно независимых векторов \(= \min\) число в порождающей системе векторов

\textbf{Теорема}

\begin{enumerate}
    \item \(\forall\) линейно независимую систему векторов в \(V\) можно дополнить до базиса пространства \(V\)

    \item из любой порождающей системы пространства \(V\) можно выделить базис пространства \(V\)
\end{enumerate}

\textbf{Доказательство}

\begin{enumerate}
    \item Пусть \(v_1, \ldots, v_m\) --- линейно независимая система

          Если \(\spann (v_1, \ldots, v_m) = V\), то \(v_1, \ldots, v_m\) --- базис

          Если \(\spann (v_1, \ldots, v_m) \subset V\), то \(\exists \ v_{m + 1} \neq 0 \in V\) и \(v_{m + 1} \notin \spann (v_1, \ldots, v_m)\)

          \(\Rightarrow v_{m + 1}\) --- не линейная комбинация остальных векторов

          \(\Rightarrow v_1, \ldots, v_{m + 1}\) --- линейно независимая система

          Повторяем рассуждения для \(v_1, \ldots, v_{m + 1}\)

          В итоге получаем \(v_1, \ldots, v_n\) --- линейно независимая система максимальная по числу элементов \(\Rightarrow v_1, \ldots, v_n\) --- базис

    \item \(\spann (v_1, \ldots, v_m) = V\), где \(v_1, \ldots, v_m\) --- порождающая система

          Если \(v_1, \ldots, v_m\) --- линейно независимая система \(\Rightarrow\) \(v_1, \ldots, v_m\) --- базис

          Если \(v_1, \ldots, v_m\) --- линейно зависимая система, то
          \[
              v_1, \ldots, v_m \xrightarrow{\text{прополка}} \underbracket{v_{j_1}, \ldots, v_{j_n}}_{\substack{\text{линейно независимая система} \\ \text{порождающая система} \\ n \le m \\ \spann (v_{j_1}, \ldots, v_{j_n}) = V}}
          \]

          \(\Rightarrow v_{j_1}, \ldots, v_{j_n}\) -- базис
\end{enumerate}

\hfill Q.E.D.

\subsection{Координаты вектора. Изоморфизм линейного пространства}

\(V\) --- линейное пространство над полем \(K (\mathbb{R}, \mathbb{C})\). \(\dim V = n\)

\(\forall x \in V: x = \sum\limits_{i = 1}^{n} \mathbf{x}_i e_i\), где \(e = (e_1, \ldots, e_n)\) --- базис в \(V\) (порождающая система)

\(\mathbf{x}_i \in K\) --- координаты вектора \(x\) относительно базиса \(e\)

\(x \in V \longrightarrow \mathbf{x} =
\begin{pmatrix}
    \mathbf{x}_{1}  \\
    \vdots \\
    \mathbf{x}_{n}
\end{pmatrix} \in K^n\), где \(
\begin{pmatrix}
    \mathbf{x}_{1}  \\
    \vdots \\
    \mathbf{x}_{n}
\end{pmatrix}
\) --- координатный столбец

\textbf{Утверждение}

\(\forall x \in V\) координаты относительно базиса \(e\) определяются единственным образом

\textbf{Доказательство}

$e$ базис $\Leftrightarrow$ порождающая линейно независимая система.

\(e_1, \ldots, e_n \Rightarrow\) порождающая система, т.е. \(x\) раскладывается на координаты

Пусть \(x = \sum\limits_{i = 1}^{n} \mathbf{x}_i e_i = \sum\limits \mathbf{x}_i' e_i\)

\(\sum\limits_{i = 1}^{n} (\mathbf{x}_i - \mathbf{x}_i') e_i = 0\) --- нулевая линейная комбинация линейно независимых векторов \(\Leftrightarrow \forall i = 1 \ldots n: \mathbf{x}_i - \mathbf{x}_i' = 0\)

\hfill Q.E.D.

\[
    x \in V \xleftrightarrow{e} \mathbf{x} \in K^n \\
\]
\[
    \text{взаимно однозначное соответствие (биекция)}
\]

\(V_1, V_2\) --- линейные пространства над одним и тем же полем $K$ называются изоморфными (\(V_1 \cong V_2\)), если между \(V_1\) и \(V_2\) существует биекция и сохраняется линейность, т.е.
\begin{gather*}
    x \in V_1 \longleftrightarrow x' \in V_2 \\
    y \in V_1 \longleftrightarrow y' \in V_2 \\
    \forall \lambda \in K: x + \lambda y \in V_1 \longleftrightarrow x' + \lambda y' \in V_2
\end{gather*}

\textbf{Свойства изоморфизма}

\begin{enumerate}
    \item \(0 \in V \longrightarrow 0' \in V'\)

          \textbf{Доказательство:}

          \(\forall \lambda \in K: \lambda x \longleftrightarrow \lambda x'\)

          Пусть \(\lambda = 0\), тогда \(0 = 0 \cdot x \longleftrightarrow 0 \cdot x' = 0'\)
          \hfill Q.E.D.

    \item \(\forall x \in V \longleftrightarrow x' \in V'\)

          \(-x \in V\) --- противоположный элемент к \(x\)

          \(-x' \in V\) --- противоположный элемент к \(x'\)

          \(\Rightarrow -x \longleftrightarrow -x'\)

          \textbf{Доказательство:}

          \(\forall \lambda \in K: \lambda x \longleftrightarrow \lambda x'\)

          Пусть \(\lambda = -1\), тогда \(-x = -1 \cdot x \longleftrightarrow -1 \cdot x' = -x'\)
          \hfill Q.E.D.

    \item
          \(x_1, \ldots, x_m \in V; x_1' \ldots x_m' \in V'\)

          \(\forall k = 1 \ldots m: x_k \longleftrightarrow x_k'\)

          \(\Rightarrow \sum\limits_{k = 1}^{m} \alpha_k x_k \in V \longleftrightarrow \sum\limits_{k = 1}^{m} \alpha_k x_k' \in V'\)

          \textbf{Доказательство:}

          По методу математической индукции
          \hfill Q.E.D.

    \item \(\underbracket{x_1, \ldots
              , x_m}_{\text{линейно независимы}} \in V \longleftrightarrow \underbracket{x_1', \ldots, x_m'}_{\text{линейно независимы}} \in V'\)

          \textbf{Доказательство:}

          \(\alpha_k \in K\)

          \(\sum\limits_{k = 1}^{m} \alpha_k x_k = 0 \longleftrightarrow \sum\limits_{k = 1}^{m} \alpha_k x_k' = 0'\)

          т.к. \(\sum\limits_{k = 1}^{m} \alpha_k x_k \longleftrightarrow \sum\limits_{k = 1}^{m} \alpha_k x_k'\) (3 свойство) и \(0 \in V \longleftrightarrow 0' \in V'\) (1 свойство)

          \(\underbracket{x_1, \ldots
              , x_m}_{\text{линейно независимы}} \in V \Leftrightarrow \forall k = 1 \ldots m: \alpha_k = 0 \Leftrightarrow \underbracket{x_1', \ldots
              , x_m'}_{\text{линейно независимы}} \in V'\)
          \hfill Q.E.D.

    \item \(\underbracket{x_1, \ldots
              , x_m}_{\text{порождающая система}} \in V \longleftrightarrow \underbracket{x_1', \ldots, x_m'}_{\text{порождающая система}} \in V'\)

          \(x_1, \ldots, x_m \in V\) --- порождающая система \(\Leftrightarrow \forall x \in V: x = \sum\limits_{k = 1}^{m} \alpha_k x_k\)

          \(\forall x \in V: x = \sum\limits_{k = 1}^{m} \alpha_k x_k \longleftrightarrow \forall x' \in V': x' = \sum\limits_{k = 1}^{m} \alpha_k x_k'\)

          т.к. \(\sum\limits_{k = 1}^{m} \alpha_k x_k \longleftrightarrow \sum\limits_{k = 1}^{m} \alpha_k x_k'\) (3 свойство) и \(x \longleftrightarrow x'\)

          \(\forall x' \in V': x' = \sum\limits_{k = 1}^{m} \alpha_k x_k' \Leftrightarrow x_1', \ldots, x_m'\) --- порождающая система
          \hfill Q.E.D.

    \item \(\underbracket{e_1, \ldots, e_n}_{\text{базис } V} \longleftrightarrow \underbracket{e_1', \ldots, e_n'}_{\text{базис } V'}\)

          \textbf{Доказательство:}

          Из свойств 4 и 5 мы знаем, что если система векторов линейно независима и порождающая, то есть это базис.
          \hfill Q.E.D.
\end{enumerate}

\textbf{Теорема}

\(V_1, V_2\) --- линейные пространства над полем \(K\)

\(V_1 \cong V_2 \Leftrightarrow \dim V_1 = \dim V_2\)

\textbf{Доказательство}

\fbox{\(\Leftarrow\)}
\(\dim V_1 = \dim V_2 \Rightarrow e_1, \ldots, e_n\) --- базис в \(V_1\) и \(e_1', \ldots, e_n'\) --- базис в \(V_2\)

Построим изоморфизм из \(V_1\) в \(V_2\)


\[
    x = \sum\limits_{i = 1}^{n} \mathbf{x}_i e_i \in V_1 \longleftrightarrow \mathbf{x} = \begin{pmatrix}
        \mathbf{x}_{1}  \\
        \vdots \\
        \mathbf{x}_{n}
    \end{pmatrix} \in K^n \longleftrightarrow x' = \sum\limits_{i = 1}^{n} \mathbf{x}_i e_i' \in V_2
\]

\[
    x \in V_1 \xleftrightarrow{\substack{\text{координатный} \\ \text{изоморфизм}}} \mathbf{x} \in K^n \xleftrightarrow{\substack{\text{координатный} \\ \text{изоморфизм}}} x' \in V_2
\]

Проверим линейность \(\forall \lambda \in K\)

\[
    x + \lambda y \longleftrightarrow \sum\limits_{i = 1}^{n} \mathbf{x}_i e_i + \lambda \sum\limits_{i = 1}^{n} \mathbf{y}_i e_i = \sum\limits_{i = 1}^{n} (\mathbf{x}_i + \lambda \mathbf{y}_i)e_i \longleftrightarrow \begin{pmatrix} 
        \mathbf{x}_{1} + \lambda \mathbf{y}_1 \\
        \vdots              \\
        \mathbf{x}_{n} + \lambda \mathbf{y}_n
    \end{pmatrix} \longleftrightarrow  \sum\limits_{i = 1}^{n}(\mathbf{x}_i + \lambda \mathbf{y}_i)e_i' =
\]

\[
    = \sum\limits_{i = 1}^{n} \mathbf{x}_i e_i' + \lambda \sum\limits_{i = 1}^{n} \mathbf{y}_i e_i' = e_i' \longleftrightarrow x' + \lambda y'
\]

\[
    x + \lambda y \longleftrightarrow x' + \lambda y'
\]

Биекция сохраняет свойство линейности \(\Leftrightarrow\) изоморфизм

\fbox{\(\Rightarrow\)}
Если \(V_1 \cong V_2\), то из 6 свойства изоморфизма мы знаем, что существует биекция между базисами этих систем \(\Rightarrow \dim V_1 = \dim V_2\)

\hfill Q.E.D.

\textbf{Следствие}

Изоморфизм конечномерных пространств --- отношение эквивалентности на множестве линейных конечномерных пространств

\(V_1 \sim V_2 \Leftrightarrow V_1 \cong V_2\)

\textbf{Доказательство}

\begin{enumerate}
    \item рефлексивность

          \(V_1 \sim V_1\), т.к. \(id_{V_1}\) --- изоморфизм

    \item симметричность

          \(V_1 \sim V_2 \Rightarrow V_2 \sim V_1\), т.к. \(\dim V_1 = \dim V_2\) по теореме выше

    \item транзитивность

          \(\begin{rcases*}
              V_1 \sim V_2 \\
              V_2 \sim V_3
          \end{rcases*} \Rightarrow\) \(V_1 \sim V_3\)

          по теореме выше

          \(\begin{rcases*}
              \dim V_1 = \dim V_2 \\
              \dim V_2 = \dim V_3
          \end{rcases*} \Rightarrow\) \(\dim V_1 = \dim V_3\)

\end{enumerate}
\hfill Q.E.D.

\newpage

\subsection{Линейное подпространство. Ранг системы векторов}

\(L \subset V\) (подмножество), если \(L\) удовлетворяет \(1-8\) аксиомам линейного пространства над полем \(K\) относительно \(+, \cdot \lambda\), то \(L\) называется линейным подпространством пространства \(V\).

\textbf{Теорема} (критерий линейного подпространства)

\(L\) --- линейное подпространство \(V \Leftrightarrow \forall x, y \in L \subset V \ \forall \lambda \in K: x + \lambda y \in L\)

(\(L\) замкнуто относительно \(+, \cdot \lambda\))

\textbf{Доказательство}

\fbox{\(\Rightarrow\)}
т.к. \(L \subset V\) и выполняются 1-8 аксиомы

\fbox{\(\Leftarrow\)}
т.к. \(L \subset V\) выполнены все аксиомы кроме 3 и 4

Пусть \(x \in L \subset V\), тогда \(x + (-1) \cdot x \in L \Rightarrow o \in L \Rightarrow \exists\) нейтральный элемент в \(L\)

Пусть \(x = 0 \in L, y \in L \Rightarrow 0 + (-1) \cdot y = -y \in L \Rightarrow \exists\) противоположный элемент

\(\Rightarrow\) для \(L\) выполнены 1-8 аксиомы линейного пространства

\hfill Q.E.D.

\textbf{Замечания}

\begin{enumerate}
    \item \(L \subset V \Rightarrow 0 \in L\)

    \item \(\dim L \le \dim V\)
\end{enumerate}

\deff{Ранг системы векторов} \(\defLeftrightarrow \dim (\spann (v_1, \ldots, v_m)) = r = \rg (v_1, \ldots, v_m)\)

\(r\) --- число \(\max\) линейно независимых векторов в \(L = \spann (v_1, \ldots, v_m)\)

по теореме о «прополке»: \(\spann (v_1, \ldots, v_m) = \spann (v_{j_1}, \ldots, v_{j_r})\) --- линейно независимы

\(v_{j_1}, \ldots, v_{j_r}\) базис \(\spann (v_1, \ldots, v_m)\) --- база системы векторов \(v_1, \ldots, v_m\)

\deff{Элементарные преобразования системы векторов:}

\begin{enumerate}
    \item удаление/добавление нулевого вектора

    \item изменение порядка векторов

    \item замена любого векторов на него же, умноженный на скаляр (\(\lambda \in K, \lambda \neq 0: v_j \rightarrow \lambda v_j\))

    \item замена любого из векторов на его сумму с любым другим вектором системы (\(v_j \rightarrow v_j + v_k\))
\end{enumerate}

\textbf{Теорема}

\(\rg (v_1, \ldots, v_m)\) не меняется при элементарных преобразованиях

\begin{enumerate}
    \item[] \textbf{\uline{Доказательство:}}
    \item[1.] Заметим, что добавление/удаление нулевого вектора никак не влияет на span, то есть на ранг.
    \item[2.] Заметим, что при перестановке у нас просто меняется порядок в разложении через эти вектора.
    \item[3.] Возьмем и умножим соответствующее $\alpha_i$ в разложении вектора на $\cfrac{1}{\lambda}$. Заметим, что <<новых>> векторов в span не добавится, и старые вектора все останутся
    \item[4.] $\ldots + a_j v_j + \ldots  + a_kv_k + \ldots = \ldots  + a_j (v_j + v_k) + \ldots  + (a_k-a_j)v_k + \ldots $. Аналогично рассуждениям из прошлого пункта получим требуемое
\end{enumerate}
\hfill Q.E.D.

\subsection{\(L_1 + L_2, L_1 \cap L_2\), формула Грассмана, \(L_1 \oplus L_2\) (прямая сумма)}

\(L_1, L_2 \in V\) --- линейные подпространства пространства \(V\)

\(L_1 + L_2 = \{ x_1 + x_2 \in V : x_1 \in L_1, x_2 \in L_2 \}\)

\(L_1 \cap L_2 = \{ x \in V : x \in L_1, x \in L_2 \}\)

\textbf{Лемма:}

Сумма и пересечение тоже линейные подпространства.

\textbf{Доказательство:}

См. критерий линейного подпространства.

\hfill Q.E.D.

\textbf{Теорема} (формула Грассмана)

\(L_1, L_2 \in V\) --- линейные подпространства пространства \(V\)

\(\dim (L_1 + L_2) = \dim (L_1) + \dim (L_2) - \dim (L_1 \cap L_2)\)

\begin{enumerate}
    \item[]  \textbf{\uline{Доказательство:}}
    \item[1.] \(\dim (L_1 \cap L_2) = 0\)

          \(L_1 \cap L_2 = \{0\}\). А что это значит? Что мы должны доказать немного другую формулу: \(\dim (L_1 + L_2) = \dim (L_1) + \dim (L_2)\)

          Возьмём $v_1,\ldots, v_n$ ---   базис $L_1$.

          Возьмём $f_1,\ldots, f_m$ ---  базис $L_2$.

          Докажем, что $v_1,\ldots,v_n,f_1,\ldots,f_m$ --- базис для $L_1+ L_2$.


          \begin{itemize}
              \item Докажем линейную независимость. От противного. Пусть лин. зависимо, тогда напишем нетривиальную линейную комбинацию:
                    $$\sum\limits_{i=1}^n \alpha_i v_i + \sum\limits_{i=1}^m \alpha_{i+m} f_i =0 \Rightarrow \underbracket{{\sum\limits_{i=1}^n\alpha_iv_i}}_{\substack{\in L_1}} = \underbracket{{{\sum\limits_{i=1}^m} -\alpha_{i+m}f_i}}_{\substack{\in L_2}}, \quad L_1 \cap L_2 = \{\mathbb{0}\} \Rightarrow$$

                    $$\Rightarrow \sum_{i=1}^n \alpha_i \underbracket{v_i}_{\substack{\text{базис}}} = \mathbb{0} = \sum_{i=1}^n -\alpha_{i+m} \underbracket{f_i}_{\substack{\text{базис}}} \Leftrightarrow \forall i: \alpha_i = 0$$
              \item Докажем порождаемость. Любой элемент суммы раскладывается (по определению) на элемент из $L_1$ и элемент из $L_2$. Откуда получили то, что нам надо.
          \end{itemize}
          Формула доказана!

    \item[2.]  \(\dim (L_1 \cap L_2) \neq 0\)
          Откуда возьмём базис пересечения: \(e_1,\ldots e_k\).

          По теореме о дополнении до базиса, т.к. \(e_1,\ldots e_k\) лежит в $L_1$ и линейно независимо, то можно дополнить до базиса \(L_1\), получим: \(e_1,\ldots e_k, v_1, \ldots v_{n-k}\)~--- базис $L_1$.

          Аналогично сделаем со вторым пространством и получим:\(e_1,\ldots e_k, f_1, \ldots f_{m-k}\) --- базис $L_2$.

          Теперь докажем, что \(e_1,\ldots,e_k,v_1,\ldots,v_{n-k}, f_1, \ldots, f_{m-k}\) --- базис суммы.

          \begin{itemize}
              \item Докажем линейную независимость. От противного. Пусть лин. зависимо, тогда напишем нетривиальную линейную комбинацию:
                    $$\sum\limits_{i=1}^{n-k} \alpha_i v_i + \sum\limits_{i=1}^{k}\alpha_{n-k+i}e_i +\sum\limits_{i=1}^{m-k}\alpha_{n+i}f_i =0$$
                    $$\underbracket{\sum\limits_{i=1}^{n-k} \alpha_i v_i + \sum\limits_{i=1}^{k}\alpha_{n-k+i}e_i}_{\substack{\in L_1}} =-\underbracket{\sum\limits_{i=1}^{m-k}\alpha_{n+i}f_i}_{\in L_2} \Rightarrow \in L_1 \cap L_2$$

                    Перенесем в другую сторону и получим, что с одной стороны у нас есть $v$ из $L_1$, с другой стороны он у нас из $L_2$. Откуда левая сумма раскладывается по векторам из $e_1$(так как он лежит в пересечении).

                    $$\sum\limits_{i=1}^{n-k} \alpha_i v_i + \sum\limits_{i=1}^{k}\alpha_{n-k+i}e_i = \sum\limits_{i=1}^{k}\beta_{i}e_i$$

                    Перенесу налево, должна получиться линейная комбинация равная нулю, а такая из-за линейной независимости может быть только тривиальной, откуда $\alpha_j$ при $v_j$ равны 0, следовательно $\alpha_j$ при всех $f_j$ и $e_k$, т.к. это базис $L_2$:

                    $$\sum_{i=1}^{m-k}\alpha_{n+i}f_i + \sum_{i=1}^k \alpha_{n-k+i}e_i = 0$$
                    
                    Откуда линейно независима.
              \item Докажем порождаемость. Любой элемент суммы раскладывается (по определению) на элемент из $L_1$ и элемент из $L_2$. Откуда разложим на базисы $L_1,L_2$ (которые указаны выше). Сложим их и получили, что данный элемент это линейная комбинация. Откуда порождаема.

              Тогда $\dim(L_1 + L_2) = n + m - k$.
          \end{itemize}

\end{enumerate}

\hfill Q.E.D.

\(L_1, \ldots, L_m \subset V\) называются дизъюнктными, если \(x_1 + \cdots + x_n = 0\), где \(x_i \in L_i, i = 1 \ldots m \Leftrightarrow \forall i = 1 \ldots m: x_i = 0\)

\(L_1 + \cdots + L_m\) называется прямой суммой, если \(L_1, \ldots, L_m\) --- дизъюнктны.

\(L_1 \oplus L_2 \oplus \ldots \oplus L_m\) --- прямая сумма линейных подпространств.

\textbf{Теорема}

\[
    L = L_1 + \cdots + L_m = \sum\limits_{k = 1}^{m} L_k, L_k \subset V
\]

\[
    L = \bigoplus_{k = 1}^{m} L_k \Leftrightarrow \text{выполнению любого из 3-х утверждений}
\]

\begin{enumerate}
    \item \(\forall j = 1 \ldots m: L_j \cap \sum\limits_{k \neq j} L_k = \{\mathbb{0}\}\)

    \item базис \(L = \) объединение базисов \(L_k\)

    \item \(\forall x \in L: \exists! x_k \in L_k: x = \sum\limits x_k\) (единственность представления элемента суммы)
\end{enumerate}

\begin{enumerate}
    \item[]\prooff{}
          1. Давайте сначала докажем из определения дизъюнктности первый пункт.

          \fbox{\(\Rightarrow\)}  Мы знаем, что $v_1+v_2+\cdots+v_m=\mathbb{0}$ возможно только если каждый из векторов --- $\mathbb{0}$. Рассмотрим $v\in L_i\cap\sum\limits_{\substack{j=1\\j\neq i}}^mL_j$. Он, как несложно заметить, лежит в $L_i$, поэтому может быть записан как $v_i$. С другой стороны, $v\in\sum\limits_{\substack{j=1\\j\neq i}}^mL_j$, что значит, что его можно записать как сумму $\sum\limits_{\substack{j=1\\j\neq i}}^mv_j$. А это значит, что $-v_i+\sum\limits_{\substack{j=1\\j\neq i}}^mv_j=\mathbb{0}$. По причине дизъюнктности, все слагаемые тут --- $\mathbb{0}$. А значит $-v_i=\mathbb{0}\Rightarrow v=\mathbb{0}$. То есть любой $v\in L_i\cap\sum\limits_{\substack{j=1\\j\neq i}}^mL_j$ является $\mathbb{0}$, что и требовалось доказать.\\

          \fbox{\(\Leftarrow\)}  Мы знаем, что $\forall i\in[1:m]~L_1\cap\sum\limits_{\substack{j=1\\j\neq i}}^mL_j=\{\mathbb{0}\}$. Хочется доказать, что $v_1+v_2+\cdots+v_m=\mathbb{0}\Leftrightarrow\forall i\in[1:m]~v_i=\mathbb{0}$. Заметим, что $v_1+v_2+\cdots+v_m=\mathbb{0}\Leftrightarrow\sum\limits_{\substack{j=1\\j\neq i}}^m v_j=-v_i$. Правая часть лежит в $\sum\limits_{\substack{j=1\\j\neq i}}^mL_j$, а левая --- в $L_i$. Это значит, что обе части лежат в их пересечении, а там лежит только $\mathbb{0}$. Значит $v_i=\mathbb{0}$. То же самое можно провести для любого $i$, получив, что все $v_i$ --- нули. Что и требовалось доказать.\\


          2. Теперь давайте докажем из определения дизъюнктности второй пункт.

          Мы знаем, что $v_1,\ldots, v_m=\mathbb{0}$ возможно только если каждый из векторов  --- $\mathbb{0}$. Рассмотрим базисы $L_i$. Возьмем все эти базисы. Очевидно они будут порождать нашу сумму. Теперь докажем линейную независимость.

          Рассмотрим нулевую линейную комбинацию объединения базисов $L_j$:

          $$\sum\limits \beta_je_i^i + \ldots + \sum\limits \beta_{f_k} e_{j_k}^k$$

          где $v_j = \sum\limits \beta_j e_i^j \in L_j$.

          Дизъюнктны $\Leftrightarrow \forall j: v_j = \mathbb{0} \Leftrightarrow \forall j:  \beta_{i_j} = 0$ т.к. $e_{i_j}^j$ базис $L_j$ $\Leftrightarrow$ объединение базисов линейно независимо.

          3. $\exists x_k$, очевидно. Докажем единственность. Пусть $x = \sum x_k = \sum x_k' \Rightarrow \sum (x_k - x'_k) = \mathbb{0} \underbracket{\Leftrightarrow}_{\substack{\text{дизъюнктность}}}\forall k: x_k - x_k' = \mathbb{0} \quad$

\end{enumerate}

\hfill Q.E.D.

\textbf{Следствие}

\(L = L_1 \oplus \ldots \oplus L_m \Leftrightarrow \dim L = \sum\limits_{i = 1}^{m} \dim L_i\)

\textbf{Доказательство}

Из п. 2.

\hfill Q.E.D.

\[
    V = \bigoplus\limits_{i = 1}^{m} L_i \Rightarrow \forall x \in V: \exists! x_i \in L_i: x = \sum\limits_{i = 1}^{m} x_i
\]

\(x_i\) --- проекция элемента \(x\) на подпространство \(L_i\) параллельно \(\sum\limits_{j \neq i} L_j\).

Если $V = L_1 \oplus L_2$, \(L_1\) --- прямое дополнение $L_2$ и наоборот.

Если $L \subset V$, то всегда $\exists L' \subset V: V = L \oplus L'$ ($L'$ выбирается неоднозначно).

\deff{Линейным} (аффинным) многообразием называется множество точек пространства \(V: P = \{x \in V: x = x_0 + l, l \in L\}\), где \(L \subset V, x_0 \in V\) (сдвинутое линейное подпространство).  Обозначается как $P = x_0 + L$.

\deff{Размерность линейного многообразия}\(\defLeftrightarrow \dim P = \dim L\)

\textbf{Теорема}

\(P_1 = x_1 + L_1; P_2 = x_2 + L_2\), где \(L_1, L_2 \subset V\) --- линейные подпространства, \(x_1, x_2 \in V\)

\[
    P_1 = P_2 \Leftrightarrow
    \begin{cases*}
        L_1 = L_2 = L \\
        x_1 - x_2 \in L
    \end{cases*}
\]
\begin{enumerate}
    \item[] \prooff{}
          \fbox{\(\Leftarrow\)}
          $\forall p_1 \in P_1 = x_1 + l_1 = x_1-x_2 + l_1 + x_2 \in P_2$

          Так как $x_1-x_2 \in L = L_2, l_1 \in  L = L_2$. Откуда $P_1\subset P_2$. Аналогично $P_2 \subset P_1$, откуда получили искомое
          \fbox{\(\Rightarrow\)}
          Посмотрим на $x_1+0$. Он лежит в $P_1$, откуда есть ему эквивалентный $x_2+l_2$из $P_2$, исходя из того, что $P_1=P_2$. Тогда $x_1-x_2$ лежит в $L_2$.

          Посмотрим на $x_2+0$. Он лежит в $P_2$, откуда есть ему эквивалентный $x_1+l_1$из $P_1$, исходя из того, что $P_1=P_2$. Тогда $x_1-x_2$ лежит в $L_1$. Откуда он лежит в пересечении.

          Теперь рассмотрим любое $l_2 \in L_2$. Ему соответсвует элемент, как $x_2 + l_2$, с другой стороны это $x_1+l_1$. Тогда $x_1-x_2+l_1 = l_2$. Откуда любой $l_2$ содержится в $l_1$. То есть $L_2 \subset L_1$. Аналогично, $L_1 \subset L_2$, откуда получили то, что нам надо.


\end{enumerate}

\hfill Q.E.D.

\textbf{Следствие}

\(P = x_0 + L\)

\(\forall x \in P \Rightarrow P_x = x + L = P\)

\textbf{Доказательство}

\begin{enumerate}
    \item \(L = L\)

    \item \(x - x_0 \in L\)
\end{enumerate}

\hfill Q.E.D.
\subsection{Фактор пространство лин. пространства}

Пусть у нас есть линейное подпространство $L$. Тогда отношение $x\sim y\Leftrightarrow x-y\in L$ является отношением эквивалентности, для любых векторов из $V$.

\deff{Факторпространство} пространства $V$ {по модулю} линейного {подпространства} $L$ $V\big|_L$ --- это фактормножество $V$ по отношению эквивалентности $\sim$ из предыдущего утверждения.

\thmm{Теорема}
$V\big|_L$ состоит из линейных многообразий на $L$.
\begin{enumerate}
    \item[] \prooff{}
          Если $x-y\in L$, то линейные многообразия $x+L$ и $y+L$ по одной из теорем ранее совпадают. То есть эквивалентные элементы порождают одинаковые многообразия.
          \vskip 0.2in
          \hfill Q.E.D.
\end{enumerate}


\thmm{Теорема}

$\dim V\big|_L=\dim V-\dim L$.
\begin{enumerate}
    \item[] \prooff{}
          Пусть $\{e_1;e_2;\ldots;e_m\}$ --- базис $L$. Дополним его до базиса $V$ векторами $\{f_1;f_2;\ldots;f_{n-m}\}$. Хочется доказать, что $\{f_1+L;f_2+L;\cdots;f_{n-m}+L\}$ --- базис $V\big|_L$.\\
          Докажем, что эта система порождающая. Нужно породить $v+L$. $v$ раскладывается по базису $\{e_1;e_2;\ldots;e_m;f_1;f_2;\ldots;f_{n-m}\}$ как $v=\sum\limits_{i=1}^m\alpha_ie_i+\sum\limits_{i=1}^{n-m}\beta_if_i$. Первая сумма лежит в $L$, то есть её можно выкинуть, многообразие останется таким же. А значит $v+L$ можно представить как $\sum\limits_{i=1}^{n-m}\beta_i(f_i+L)$, ведь по определению суммы многообразий это $\left(\sum\limits_{i=1}^{n-m}\beta_if_i\right)+L$.\\
          Теперь докажем линейную независимость. Рассмотрим нулевую линейную комбинацию $\sum\limits_{i=1}^{n-m}\beta_i(f_i+L)$. Она, как мы уже знаем, равна $\left(\sum\limits_{i=1}^{n-m}\beta_if_i\right)+L$. Это должно быть равно нейтральному элементу (то есть $L$). Когда эти линейные многообразия равны? Когда $\sum\limits_{i=1}^{n-m}\beta_if_i\in L$. То есть $\sum\limits_{i=1}^{n-m}\beta_if_i-\sum\limits_{i=1}^{m}\alpha_ie_i=0$. Но это же линейная комбинация векторов подсистемы $\{e_1;e_2;\ldots;e_m;f_1;f_2;\ldots;f_{n-m}\}$, а значит она линейно независима. А значит $\forall i\in[1:n-m]~\beta_i=0$, что значит, что линейная комбинация $\sum\limits_{i=1}^{n-m}\beta_i(f_i+L)$ тривиальна.
          
          \hfill Q.E.D.
\end{enumerate}



