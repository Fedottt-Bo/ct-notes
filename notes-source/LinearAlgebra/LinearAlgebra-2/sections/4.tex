\subsection{Сопряженный оператор.}

\deff{def:} $\mathcal{A}\in L(U,V)$. $\mathcal{A}^*: V^*\xrightarrow{} U^*$:

$\forall f \in V^*: \forall x \in U: (\mathcal{A}^*f)(x) = f(\mathcal{A}x) $

$\mathcal{A}^*$ называется \deff{линейным отображением, сопряженным к $\mathcal{A}$}.

Покажем, что $A^* \in L(V^*,U^*)$. 

$$\forall x \in U:\forall \lambda \in K: \forall f_1,f_2 \in V^*: \mathcal{A}^*(\lambda f_1+f_2)= (\lambda f_1 + f_2)(\mathcal{A}x) = \lambda f_1(\mathcal{A}x)+ f_2(\mathcal{A}x)$$
Откуда линейное отображение.

Если $U \xrightarrow{} V, \mathcal{A} \in End(V): \mathcal{A}^*\in End(V^*)$ ----  линейный оператор сопряженный к $\mathcal{A}$ или \deff{сопряженный оператор}.

Пусть $V$ это пространство со скалярным произведением (евклидово или унитарное).

По теореме Рисса: $\forall f \in V^*: \exists! y \in V: \forall x \in  V: f(x) = (x,y)$.

$\mathcal{A} \in End(V)$. Пусть $g = \mathcal{A}^* f \in V^*$, тк $\mathcal{A}^* \in End(V^*)$. Тогда по теореме Рисса $\exists! z \in V: \forall  x \in V: g(x) = (x,z)$.

Тогда $g = \mathcal{A}^* f \xleftrightarrow[\text{взаимоодназначно}]{\text{по т. Рисса}} z = \mathcal{A}^* y$.

\deff{def:} $\mathcal{A}^*: V \xrightarrow{} V: \mathcal{A^*}\in End(V): \forall y \in V: \mathcal{A}^*y = z$

$(x, \mathcal{A}^* y)=(x,z)=g(x)=(\mathcal{A}^* f )(x) = f(\mathcal{A}x) = (\mathcal{A}x,y)$

\deff{def:} $A \in End(V)$. $A^* \in End(V)$ называется сопряженным оператором к $\mathcal{A}$, если $\forall x,y \in V: (\mathcal{Ax,y) = (x,\mathcal{A}^*y)}$.

Покажем линейность нашего отображения: $$\forall \lambda \in L, \forall y_1,y_2 \in V: (x, \mathcal{A}^*(\lambda y_1 + y_2))= (\mathcal{A}x, \lambda y_1 + y_2) = \overline{\lambda} (\mathcal{A}x, y_1) + (\mathcal{A}x, y_2) = (x, \lambda \mathcal{A}^* y_1 + \mathcal{A}^* y_2)$$

\textbf{Свойства $\mathcal{A}$:}

\begin{enumerate}
    \item $(V, (\cdot, \cdot)), e= (e_1,\ldots, e_n)$ - базис.

$$A^\circledast = \overline{\Gamma^{-1}} A^* \overline{\Gamma}$$
где $A^* = \overline{A^T}$

\textbf{Доказательство:}

$x \xleftrightarrow{e}x, y \xleftrightarrow{e}y$, получаю, что:
$$\forall x,y \in V: (\mathcal{A}x,y) = (x, \mathcal{A}^* y) \Leftrightarrow (\mathcal{A}x)^T \Gamma\, \overline{y} = x^T \Gamma \overline{A^\circledast y} $$
$$x^T (A^T \Gamma)\overline{y}=x^T \Gamma \overline{A^\circledast}\overline{y}$$
\hfill Q.E.D.

\item $(\mathcal{A}^*)^* = \mathcal{\mathcal{A}} $.
\item  $(\mathcal{A} + \lambda B)^* = \mathcal{A}^* +\overline{ \lambda} B^*$

\textbf{Доказательство:}

$\forall \mathcal{A}, \mathcal{B}\in End(V):\forall \lambda \in K:\forall x,y \in V: (x, (\mathcal{A } + \lambda \mathcal{B})^* y) = ((\mathcal{A} + \lambda \mathcal{B})x, y)= (x, (\mathcal{A}^* +\overline{\lambda}\mathcal{B}^* )y)$

\hfill Q.E.D.

\item $\mathcal{A}, \mathcal{B}\in End(V) \Rightarrow (\mathcal{A}\mathcal{B})^* = \mathcal{B}^* \mathcal{A}^*$

Доказывается аналогично прошлому пункту.

\item $\ker \mathcal{A}^* = (\Im \mathcal{A})^\perp$ и $\Im \mathcal{A}^* = (\ker \mathcal{A})^\perp$

\textbf{Доказательство:}

2-ое равенство:

$\forall x \in \ker \mathcal{A}: \mathcal{A}x = \zero : \forall y \in V: (x, \mathcal{A}^*y) = (\mathcal{A}x,y)=0$

$\Rightarrow \ker \mathcal{A} \subseteq (\Im \mathcal{A}^*)^\perp$ по теореме о ранге и дефекте: $\rg \mathcal{A} + def \mathcal{A} = n$. Хочу показать, что $\rg \mathcal{A} = \rg \mathcal{A}^*$. $\rg A^\circledast = \rg A^* = \rg A$. А это значит, что $\rg \mathcal{A}^* + def \mathcal{A} = n$: $\dim (\Im \mathcal{A}^*) + \dim (\ker \mathcal{A}) = n$.
Тогда получается, что размерности совпадают.

1-ое равенство:

$\ker A^* = (\Im \mathcal{A}^*)^\perp$ - напишем для $V^*$ первым равенством  и воспользуемся вторым свойством.

\hfill Q.E.D.

\item $\varepsilon^* = \varepsilon$. Если $\mathcal{A}$ невырожденное $\exists \mathcal{A}^{-1} \Rightarrow \mathcal{A}^*$ невырожденно, $(\mathcal{A}^*)^{-1} = (\mathcal{A}^{-1})^*$.

\textbf{Доказательство:}

$\mathcal{A}$ невырожденно $\Leftrightarrow \ker \mathcal{A} = \{\zero\}$ $\Leftrightarrow A^*$ невырожденно.

Посмотрим на $\mathcal{A}\mathcal{A^{-1}}$. $\mathcal{A}\mathcal{A}^{-1}= \mathcal{A}^{-1}\mathcal{A} = \varepsilon$. Так же:
$(\mathcal{A}\mathcal{A}^{-1})^*= (\mathcal{A}^{-1}\mathcal{A})^* = \varepsilon^* = \varepsilon$

Раскроем скобки по 4-ому свойству и получим, что нам надо.

\hfill Q.E.D.

\item $\chi_\mathcal{A}(\lambda) = 0  \Leftrightarrow \chi_{\mathcal{A}^*}(\overline{\lambda})=0$. А также $\chi_{\mathcal{A}^*}(t)=\overline{\chi_\mathcal{A}(\overline{t})}$.

\textbf{Доказательство:}

$e$ о.н.б. $V$. $\mathcal{A} \xleftrightarrow{e}A, \mathcal{A}^* \xleftrightarrow{e}A^*:$
$$0 = \overline{ 0} = \overline{\chi_\mathcal{A}(\lambda)}= \overline{\det (A - \lambda E)} = \det (\overline{A^T} - \overline{\lambda} E) = \chi_{\mathcal{A}^*} (\overline{\lambda})$$
\hfill Q.E.D.

\item $\lambda$ - с.ч. $u$ - с.в. $\mathcal{A}$, $\mu$ - с.ч. $v$ - с.в. $\mathcal{A^*}, \lambda \neq \overline{\mu} \Rightarrow u \perp v$.

\textbf{Доказательство:}

$\lambda(u,v)=(\lambda u,v)=(\mathcal{A}u, v ) = (u,\mathcal{A}^* v) = (u, \mu v) = \overline{\mu}(u,v)$. Откуда:

$(\lambda - \overline{\mu})(u-v) = 0$. Откуда уже и требуется то, что нам надо

\hfill Q.E.D.

\item $L \subset V$ инвариантно относительно $\mathcal{A} \Rightarrow L^\perp$ инвариатно относительно $\mathcal{A}^*$  

\textbf{Доказательство:}

$x\in L: y\in L^\perp: (x,y) =0$.

$(x,\mathcal{A}^*y) = (\mathcal{A}x, y ) = 0 \Rightarrow \mathcal{A}^*y \in L^\perp$

\hfill Q.E.D.
\end{enumerate}

\newpage
\subsection{Нормальные операторы.}

$\mathcal{A}\in End(V)$ называется \deff{нормальным оператором}, если $\mathcal{A}\mathcal{A}^* = \mathcal{A}^* \mathcal{A}$ - перестановочны:

$$\Leftrightarrow \forall x,y \in V: (\mathcal{A}x, \mathcal{A}y) = (\mathcal{A}^* x, \mathcal{A}y)$$

\textbf{Свойства нормальных операторов.}

\begin{enumerate}
    \item $\mathcal{A}$ нормальный оператор $\Leftrightarrow$ в некотором о.н.б. e :$A A^* = A^*A$.

    \textbf{Доказательство:}

    В правую сторону очевидно, теперь в левую сторону. Пусть $e$ о.н.б $A A^* =A^* A$. Пусть $e'$ другой базис и $T_{e\rightarrow{}e'}$. $A =\xleftrightarrow[]{e'}A', A^* \xleftrightarrow[]{e'}A'^*$

    $A' = T^{-1}AT, A'^* = T^{-1}\overline{A^T}T$

    Покажем, что $A'A'^* = A'^*A':$
    $$A'A'^* = T^{-1}A A^*T = T^{-1}A^*AT = A'^*\cdot A'$$
    Откуда $\mathcal{A}$ - нормальный.
    
    \hfill Q.E.D.

    \item $\ker \mathcal{A} = \ker A^*,(\ker \mathcal{A})^\perp = \Im \mathcal{A}$. $\ker (\mathcal{A}^2)=\ker (\mathcal{A})$.

    \textbf{Доказательство:}

    \begin{enumerate}
        \item $x\in \ker A \Leftrightarrow \mathcal{A}x = \zero \Leftrightarrow (\mathcal{A}x, \mathcal{A}x) = 0 \Leftrightarrow (\mathcal{A}^*x, \mathcal{A}^*x) = 0 \Leftrightarrow \mathcal{A}^*x = \zero \Leftrightarrow x\in \ker A^*$

        \item очевидно, из свойства 5 для сопряженного оператора.
        \item $x\in \ker \mathcal{A}\subseteq \mathcal{\ker A}^2$.

        Пусть $x \in \ker \mathcal{A}^2 \Leftrightarrow (\mathcal{A}^2 x, \mathcal{A}^2x) = 0 \Leftrightarrow (\mathcal{A}x, \mathcal{A}^* \mathcal{A}^2x) = 0 \Leftrightarrow (\mathcal{A}x, \mathcal{A}(\mathcal{A}^*\mathcal{A}x)) = 0 \Leftrightarrow (\mathcal{A^*(\mathcal{Ax}),\mathcal{A}^*(\mathcal{A}x)})$.
        $\Leftrightarrow \mathcal{A}x \in \ker \mathcal{A}^* =\ker \mathcal{A} \Leftrightarrow \mathcal{A}x = \zero \Leftrightarrow x \in \ker \mathcal{A}$
    \end{enumerate}

    \hfill Q.E.D

    \item   $B = \mathcal{A}-\lambda \varepsilon$ нормальный оператор, $\forall \lambda \in K$.

    \textbf{Доказательство:}

    $B^* = (\mathcal{A} - \lambda \varepsilon)^* = \mathcal{A}-\overline{\lambda}\varepsilon$

    $BB^* = (\mathcal{A}-\lambda \varepsilon) (A^* - \overline{\lambda} \varepsilon) =B^* \cdot B$, так как это многочлены от $A$ - перестановочные.

    \hfill Q.E.D.

    \item $\lambda$ с.ч.$v$ - с.в. $\Leftrightarrow \overline{\lambda}$ с.ч $v$ с.в $\mathcal{A}^*$

    \textbf{Доказательство:}

    $\lambda$ с.ч. $v$ с.в. $\mathcal{A} \Leftrightarrow v \neq 0 \in \ker (\mathcal{A}-\lambda \varepsilon) = \ker B = \ker B^* \Leftrightarrow v \in \ker B^* \Leftrightarrow v\in \ker(A^*-\overline{\lambda}\varepsilon) \Leftrightarrow \mathcal{A}^*v = \overline{\lambda} v$
   
    \hfill Q.E.D.

    \item $\lambda$ с.ч, $v$ с.в $\mathcal{A}$, а также $\mu$ с.ч., а $u$ его с.в. $\mathcal{A}$. Если $\lambda\neq \mu \Rightarrow u \perp v$.

    \textbf{Доказательство:}

    $\mu(u,v)=(\mu u, v)=(\mathcal{A}u,v)=(u,\mathcal{A}^*v) = (u,\overline{\lambda}v)=\lambda(u,v)$, откуда $(u,v)$ - то, что нам и требовалось.

    \hfill Q.E.D.

    
\end{enumerate}

\thmm{Теорема(о каноническом виде матрицы нормального оператора в унит. пространстве)} 

$\mathcal{A} \in End(V)$. 

$\mathcal{A}$ нормальный оператор $\Leftrightarrow \exists$ о.н.б. $e$ пространства $V$, $\mathcal{A}\xleftrightarrow[]{e}\Lambda= diag(\lambda_1,\ldots,\lambda_n)$, $\lambda_i \in K$

\textbf{Доказательство:}

В левую сторону. $\mathcal{A} \xleftrightarrow[\text{о.н.б}]{e}\Lambda = diag (\lambda_1,\ldots,\lambda_n)$ по условию, а также  $\mathcal{A}^* \xleftrightarrow[\text{о.н.б}]{e}\Lambda^* = diag (\overline{\lambda_1},\ldots,\overline{\lambda_n})$ (см. замечание). 

Откуда $\Lambda \cdot \Lambda^* = \Lambda^* \Lambda$ в некотором о.н.б $\Rightarrow$ по первому свойству нормальный оператор.

Докажем в правую сторону. Пусть $\lambda_1$ корень $\chi_{\mathcal{A}}$, тк мы находимся сейчас в унитарном пространстве, то $\lambda_1$ - собственное число. Пусть $v_1$ его собственный вектор, нормированный. Заметим, что тогда $\overline{\lambda_1}$ с.ч. $v_1$ с.в. $\mathcal{A}^*$.

Пусть $L = \span(v_1). V  = L \oplus L^\perp$. Мы знаем, что $L$ инвариантно относительно $\mathcal{A}$, а также $L$ инвариантно относительно $\mathcal{A}^*$. Из свойства 9 сопряж. оператора, $L^\perp$ инвариантно относительно $A^*$, а также $L^\perp$ инвариантно относительно $\mathcal{A}$. Откуда матрицу можно в блочно-ступенчатый вид из 2 блоков.

Воспользуемся методом математической индукции:

\begin{enumerate}
    \item База $n=1$ очевидно.
    \item Пусть верно для $n=k$, что матрица будет иметь диаг. вид.

    Докажем, для $n  =k+1$. $\dim V = k+1$. Пусть $\lambda_1$ с.ч. $v_1$ соотв ей нормированный с.в. $\mathcal{A}$. Пусть $L = \span (v_1)$. $V = L \oplus L^\perp$, $\dim L^\perp = k$. По индукции я знаю, что $\mathcal{A}\Big|_{L^\perp} \xleftrightarrow{v_2,\ldots,v_{k+1} \text{ - о.н.б.}}\begin{pmatrix}
        \lambda_2 & & \\
        & \ddots &\\
        && \lambda_{k+1}
    \end{pmatrix}$

    Мы знаем $(v_1,\ldots, v_{k+1})$, что это будет о.н.б, откуда уже очевидно следует требуемое нами.
\end{enumerate}

\hfill Q.E.D.

\textbf{Замечания:}
\begin{enumerate}
    \item Очевидно, что $\mathcal{A}^* \xleftrightarrow[\text{о.н.б.}]{e} \overline{\Lambda^T}=\Lambda^* = diag (\overline{\lambda_1},\ldots, \overline{\lambda_n})$
    \item Очевидно, что $\lambda_j$ с.ч. $\mathcal{A}$ ( $\overline{\lambda_j}$ с.ч. $\mathcal{A}^*$)
    \item Очевидно, что $\mathcal{A}$ - норм. $\Rightarrow$ о.п.с.
\end{enumerate}

\textbf{Следствие 1.} $\mathcal{A}$  нормальный оператор в унитарном пространстве $\Leftrightarrow V = \bigoplus\limits_{\lambda}V_\lambda,$ причем  $V_{\lambda}\perp V_{\mu}$ $(\lambda \neq \mu)$.

\textbf{Следствие 2.} $\forall A A^* =A^*A$. $\exists T$ унитарная $(T^* = T^{-1})$, где $T^* AT  = \Lambda$.

\thmm{Теорема (о каноническом виде матрицы нормального оператора в евкл. пространстве)}

$\mathcal{A} \in End(V)$, V - евклидово.

$\mathcal{A}$ норм. $\Leftrightarrow \exists $ о.н.б. $e$. $\mathcal{A}\xleftrightarrow[e]{}\Lambda = \begin{pmatrix}
    \lambda_1 & & & & &0\\
    & \ddots & & & &\\
    & & \lambda_k & & &\\
    & & & \Phi_1 & &\\
    & & & & \ddots & \\
    0& & & & & \Phi_m
\end{pmatrix}$   имеет вот такой блочно-диагональный вид.

Причем $\lambda_i \in \R$ -  собственные числа (повторяются с учетом кратности), а каждое $\Phi_j$ имеет вид $\begin{pmatrix}
    \alpha_j & \beta_j \\
    - \beta_j & \alpha_j
\end{pmatrix}$, где $\alpha_j \pm i \beta_j$ - пара сопряженных корней $\chi_{\mathcal{A}}(t)$ (повторяются с учетом кратности).

\textbf{Замечание:} $\mathcal{A}^* \xleftrightarrow{\text{e - о.н.б.}} \Lambda^* = \Lambda^T$.

Доказательство будет дальше, но чтобы нам это доказать, сперва позамечаем куда более интересные факты:

Пусть $V_\C$ это комплесификация $V$. $\mathcal{A} \in End(V): \mathcal{A}_\C$ это продолжение $\mathcal{A}$ на $V_\C$

$(V,(\cdot,\cdot)_\R)$ и есть теперь $(V_\C, (\cdot,\cdot)_\C)$.

Введем скалярное произведение на $V_\C$ таким образом:

$\forall z_1,z_2 \in V_\C: (z_1,z_2)_\C = (x_1+y_1i, x_2 + y_2i)_\C =(x_1 +x_2)_\R+ (y_1,y_2)_\R + i ((y_1,x_2)_\R - (x_1,y_2)_\R) \in \C$.

\textbf{Свойства $(\cdot, \cdot)_\C$}:

\begin{enumerate}
    \item $\forall x,y \in V: (x,y)_\C = (x,y)_\R$
    \item $\overline{(z_1,z_2)}_\C = (\overline{z_1},\overline{z_2})_\C$
    \item $(z,\overline{z}) = 0 \Leftrightarrow \begin{cases}
        ||x|| =||y|| \\
        (x,y) = 0
    \end{cases}, x = Re z, y = \Im z$

    \textbf{Доказательство:}

    $0 = (z,\overline{z})= ||x||^2 -||y||^2 + i((y,x) + (x+y))$

    \hfill Q.E.D.
\end{enumerate}




\textbf{Свойства $\mathcal{A}_\C$}:

$\forall z \in V_\C, \mathcal{A}_\C z=\mathcal{A}x + i \mathcal{A}y$

\begin{enumerate}
    \item $(\mathcal{A}\mathcal{B})_\C = \mathcal{A}_\C \mathcal{B}_\C$
    \item $\mathcal{A}$ -  невырожденная $\Rightarrow$ $\mathcal{A}_\C$ невырожденная и $(\mathcal{A}_\C)^{-1} = (\mathcal{A}^{-1})_\C$
    \item $(\mathcal{A}_\C)^* = (\mathcal{A}^*)_\C$

    \item $\mathcal{A}$ норм. $\Rightarrow \mathcal{A}_\C$ норм.
    \item $\lambda$ с.ч. $\mathcal{A}$, $V_{\lambda}$ собственное подпространство $\mathcal{A} \Rightarrow (V_{\lambda})_{\C}$ собственное подпространство $\mathcal{A}_\C$, $\lambda$ с.ч. $\mathcal{A}_\C$. 
\end{enumerate}

\textbf{Доказательство теоремы:}

\deff{Давайте сначала докажем достаточность} $(\Leftarrow):$

$\Lambda = \begin{pmatrix}
    \lambda_1 & & & & &0\\
    & \ddots & & & &\\
    & & \lambda_k & & &\\
    & & & \Phi_1 & &\\
    & & & & \ddots & \\
    0& & & & & \Phi_m
\end{pmatrix} \Leftrightarrow\Lambda^* = \Lambda^T = \begin{pmatrix}
    \lambda_1 & & & & &0\\
    & \ddots & & & &\\
    & & \lambda_k & & &\\
    & & & \Phi_1^T & &\\
    & & & & \ddots & \\
    0& & & & & \Phi_m^T
\end{pmatrix}$ 

Давайте умножим матрицу $\Lambda$ и $\Lambda^*$. Для этого сначала умножу, $$\Phi_j^T \cdot \Phi_j = \begin{pmatrix}
    \alpha_j &-\beta_j\\
    \beta_j & \alpha_j
\end{pmatrix} \begin{pmatrix}
    \alpha_j & \beta_j \\
    -\beta_j & \alpha_j
\end{pmatrix} = \Phi_j \Phi_j^T$$ Откуда при произведении $\Lambda \cdot \Lambda^* = \Lambda^* \cdot \Lambda$, что по первому свойству нормального оператора говорит о том, что $\mathcal{A}$ - нормальный.

\deff{Теперь докажем в другую сторону. }

Пусть $\mathcal{A}$ нормальный оператор. Комплесифицируем и получим $V \rightarrow{} V_\C, \mathcal{A}\rightarrow{} \mathcal{A}_\C$.  

Наша цель: построить базис из вещ. векторов(т.е. $\in V$), в котором матрица $\mathcal{A}_\C$ будет иметь заявленный вид, потому что по свойству матрица $\mathcal{A}$ будет иметь такой же вид.

$\mathcal{A}_\C$ нормальный. Воспользуемся теоремой о канонич. виде матрицы нормального оператора для евклидова пространства.

Из нее существует о.н.б $V_\C$ нормированный и наше $V_\C$ представляется как:
$$V_\C = \bigoplus\limits_{\lambda \in \R} V_\lambda^\C\bigoplus\limits_{\mu_1,\mu_2 \in \C}V_\mu^\C$$
$\lambda$ - рациональные корни $\chi_{\mathcal{A}_\C}$, $\mu_1,\mu_2$ - парные корни $\in C$.

$\lambda \in \R: V_\lambda^\C = (V_\lambda)_\C$. Причем как мы помним, $V_\lambda^\C \perp V_\mu^\C$. То есть $(V_\lambda)_\C = (\span(w_1,\ldots,w_k))_\C$, где $w_1,\ldots,w_k  \in V$ - ортонормированные и вещественные. Теперь посмотрим на вторые  виды векторов:
$$V_{\mu_1,\,u_2}^\C = \bigoplus\limits_j \span_\C (z_j, \overline{z_j}) = \bigoplus\limits_j(\span_\R(u_i,u_j))_\C$$
Пусть $\alpha + i\beta = \mu_1 = \overline{\mu_2}$, $z$ - собственный вектор $\mu_1$, а $\overline{z}$ с.в.  $\overline{\mathcal{A}_\C z} = \mathcal{A}_\C \overline{z}$.
$$\mathcal{A}_\C \overline{z} = \overline{\mathcal{A}_\C z} = \overline{\mu_1 z} = \mu_2\overline{z}$$
Пусть $u_j =\cfrac{z_j + \overline{z_j}}{2}, v_i = \cfrac{z_j -\overline{z_j}}{2i}$. $\span (z_j, \overline{z_j}) = \span(u_i,u_j)$ в таком случае.

Как мы знаем $V_{\mu_1}^\C\perp V_{\mu_2}^\C: z \perp \overline{z} \Rightarrow \begin{cases}
    ||u || = ||v||\\
    (u,v) = 0
\end{cases}$.

Откуда $V_\C = \span(w_1,\ldots,w_k, u_1,v_1,\ldots,u_m,v_m)_\C$ - ортогональный базис.

\sout{По идее тут должны быть все $w_i$ для каждого с.ч., мб кучерук неверно написала выше про $w_1,\ldots,w_k$? Может там на самом деле о.н.б для всех $\lambda\in\R$}

Базис у нас получается состоит из вещественных векторов $\Rightarrow \mathcal{A}_\C\leftrightarrow \Lambda$ вещественный. 

Тогда посмотрим на матрицу оператора в нашем ортогональном базисе. 
$$\mathcal{A}_\C w_j = \lambda_j w_j$$
$$ \mathcal{A}_\C u = A_\C (\cfrac{z + \overline{z}}{2}) =\alpha (\cfrac{z+\overline{z}}{2}) + i^2 \beta (\cfrac{z-\overline{z}}{2i})  = au - \beta v$$
Аналогично:
$$\mathcal{A}_\C v = av + \beta u$$
Заметим, что мы получили нужный нам вид матрицы, но мы еще не отнормировали вторую половину векторов!

Как мы знаем: $||z|| = 1 =||u||^2 + ||v ||^2 = 2 ||u||^2 = 1$. Откуда, чтобы отнормировать вектора, я должен умножить все вектора из второй части на $\sqrt{2}$. Получу такой же вид матрицы, откуда теорема доказана!

\hfill Q.E.D.

\textbf{Следствие:} $\forall AA^T =A^T A$, $A$ - норм. вещ матрица. Тогда существует ортогональная $T (T^{-1}=T^T)$, такая что $T^TAT = \Lambda$ из теоремы.






\newpage
\subsection{Самосопряжение и изометрические операторы.}


\deff{def:} $\mathcal{A}\in End(V)$. $\mathcal{A}$ называется \deff{самосопряженным} если $\mathcal{A}^* =\mathcal{A} \Leftrightarrow \forall x,y \in V: (\mathcal{A}x,y) = (x,\mathcal{A}y) $

В $\R$ они будут называться \deff{симметричными}. В $\C$ они будут названы \deff{эрмитовы}. 

Очевидно $\mathcal{A}$ самосопряженный $\Rightarrow$ нормированы.

\textbf{Свойства:}

\begin{enumerate}
    \item $\mathcal{A}$ самосопряж. $\Leftrightarrow \exists $ о.н.б., где $A = A^*$. 
    \item $\mathcal{A}, \mathcal{B}$ самосопряж $\Rightarrow (\mathcal{A}+\lambda \mathcal{B})$ самосопряж, $\forall \lambda \in \R$
    \item $\mathcal{A},\mathcal{B}$ самосопряж. и $\mathcal{A}\mathcal{B}=\mathcal{BA}\Rightarrow \mathcal{AB}$ самосопряженный.
    \item $\exists \mathcal{A}^{-1},\mathcal{A}$ самосоп. $\Rightarrow \mathcal{A}^{-1}$ самосопряж.
    \item $\mathcal{A}$ самосопряж. $\Leftrightarrow \begin{cases}
        \mathcal{A} \text{ - норм.}\\
        \text{все корни $\chi_{\mathcal{A}}$ веществ.   }
    \end{cases} \Leftrightarrow$ о.п.с. с вещ. собств числами $V_\lambda \perp V_\mu$, при $\lambda\neq \mu$

    \textbf{Доказательство:}

    \begin{enumerate}
        \item теорема о каноноч. виде матрицы нормального оператора в унитарном пространстве. $\Lambda = diag (\lambda_1,\ldots, \lambda_n)$, $\Lambda^* = diag(\overline{\lambda_1},\ldots,\overline{\lambda_n})$. $\mathcal{A}$ самосопряж. $\Leftrightarrow A = A^* \Leftrightarrow \Lambda = \Lambda^* \Leftrightarrow \lambda_j = \overline{\lambda_j}$
        \item пользуемся теоремой о каноническом виде матрицы нормального оператора в евклид. пространстве, получаем $\Phi_j = \Phi_j^T$, то есть комплексных корней нет.
    \end{enumerate}

    \hfill Q.E.D.


    \item $L \subset V$ лин. подпространство инвариантно относительно $\mathcal{A} \Rightarrow L^\perp$ инвариантно относительно $A$

    Очевидно из свойств сопряж. оператора
\end{enumerate}


\deff{def:} $Q\in End(V)$. $Q$ называется \deff{изометрическим},если $Q$ невырождена  и $Q^*=Q^{-1} \Leftrightarrow \forall x,y \in V: (Qx,Qy)= (x,y)$. 

\textbf{Замечание:}  $(Qx,Qy) = (Q^*Qx,y) \rightarrow{ Q \cdot Q^* = \varepsilon}$

\textbf{Свойства:}

\begin{enumerate}
    \item $Q$ изометрич. $\Leftrightarrow \exists $ о.н.б. $Q^{-1}=Q^*$.
    \item $Q$ изометрич. $\Leftrightarrow Q$ переводит о.н.б. в о.н.б.
    \textbf{Доказательство:}

    \deff{В правую сторону:} $Q$ изометрич., $e$ о.н.б.: $(e_i,e_j) = \delta_{i,j}$. Откуда если просто расписать определение, то получим $(e_i,e_j) = (Q e_i,Qe_j) = \delta_{ij}$

    \deff{В обратную сторону:} $Qe_i = e_i'$, $e$ - о.н.б, $e'$ о.н.б.

    $\forall x,y \in V: (Qx,Qy) = \sum\limits_{i}\sum\limits_{j}x_i\overline{y}_j (e_i',e_j') = \sum\limits_{i}\sum\limits_{j}x_i\overline{y}_j \delta_{ij}= (x,y)$, откуда изометрический.

    \hfill Q.E.D.

    \item $Q,R$ изометрич. $\Rightarrow QR$ изометрич.
    \item $Q$ изометрич. $\Rightarrow Q^{-1}$ изометрич.
    \item $Q$ изометрич. $\Leftrightarrow \begin{cases}
        Q \text{ - норм.}\\
        \text{все корни по модулю равны 1}
    \end{cases}$

    Доказательство аналогично пункту 5 из сопряж.
    \item  $L \subset V$ инвариантно относительно $Q \Rightarrow L^\perp$ инвариантно относительно $Q$

    \textbf{Доказательство:}

    $\forall x\in L: \forall y \in L^\perp$. $Q$ - невырожд. оператор

    todo слава умер от желания спать
    

    \hfill Q.E.D.
    
\end{enumerate}

