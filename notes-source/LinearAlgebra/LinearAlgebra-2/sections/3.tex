\subsection{Основные определения.}

Начнем с определений. 

\deff{def:} $V$ - линейное пространство над полем $\mathbb{R}$.

$(\cdot, \cdot ): V\times V \xrightarrow{} \R$ называется \deff{скалярное произведение}, если она удовлетворяет 4-ем аксиомам.

$\forall x,y \in V$, $\forall \lambda \in \R$:

\begin{enumerate}
    \item $(x,y) = (y,x)$ --- симметричность
    \item $(x_1 +x_2,y) = (x_1,y)+(x_2,y)$
    \item $(\lambda x, y) = \lambda (x,y)$
    \item $\forall x \neq 0: (x,x)>0$
\end{enumerate}

\deff{def:} $(V, (\cdot, \cdot))$ $\dim V =n<\infty$ называется \deff{евклидовым пространством} или вещественным линейным пространством со скалярным произведением.

\textbf{Замечание:} Если $V$ бесконечномерно, то это называется \deff{гильбертовым пространством}

\deff{def:} $V$ - линейное пространство над полем $\C$.

$(\cdot, \cdot): V\times V \xrightarrow{} \C$ функция называется \deff{псевдоскалярным} пространством.

\begin{enumerate}
    \item $(x,y) = \overline{(y,x)}$ --- симметричность
    \item $(x_1 +x_2,y) = (x_1,y)+(x_2,y)$
    \item $(\lambda x, y) = \lambda (x,y)$
    \item $\forall x \neq 0: (x,x)>0$
\end{enumerate}

Такая функция называется \deff{полуторалинейной}.

\deff{def:} $\dim V = n < \infty$, $(V, (\cdot, \cdot))$ называется \deff{унитарным} пространством или эрмитовый или псевдоевклидовой или комплексным линейным пространством с псевдоскаляром.

\textbf{Замечание:} Если вы не напишите слово вещественные или комплексные в работе или на экзамене, то вам инста бан.

\deff{def:} Введем норму: $\forall x \in V: ||x|| =\sqrt{(x,x)}$ --- \deff{евклидова норма}.

Давайте проверим выполняемость свойств нормы.

\begin{enumerate} %todo
    \item $\forall x \neq 0 \Rightarrow ||x|| \neq 0$ (невырожденность) --- выполнена.
    \item  $\forall \lambda \in K \Rightarrow ||\lambda x|| = \sqrt{(\lambda x, \lambda x)} = |\lambda|\cdot||x||$ (однородность) --- выполнено.
    \item  $\forall x, y\in V$ неравенство треугольника. 
    $$||x+y||\leq ||x|| + ||y||$$
    Мы будем пользоваться в доказательстве неравенством КБШ. Его доказательство вы можете найти в конспекте первого семестра по матанализу. 
    $$||x+y||^2 = (x+y,x+y) = ||x||^2 + (x,y) + (y,x) + ||y||^2 \leq ||x||^2 + 2 Re(x,y) + ||y||^2 \leq$$
    $$\leq ||x||^2 + 2||x||\cdot||y|| + ||y||^2 \leq  (||x|| + ||y||)^2$$
    $$\Rightarrow ||x+y|| \leq ||x|| + ||y||$$
\end{enumerate} 

\deff{def:}  $\forall x \in V:$ $||x||$ - \deff{длина вектора}. $\varphi$ называем углом между $x$ и $y$, таким, что $\cos \varphi = \cfrac{(x,y)}{||x||||y||}$

\textbf{Пример:}

\begin{enumerate}
    \item Возьмем $\mathbb{R}^n$. $\forall x,y \in \C^n$. $(x,y) = \sum\limits_{i=1}^n x_i \overline{y_i}$. Заметим что выполнены все 4 аксиомы скалярного произведения
    $$||x|| = (\sum\limits_{i=1}^n |x_i|^2)^{1/2} = \sqrt{(x,x)} = \sqrt{\sum\limits_{i=1}^n x_i \overline{x_i}}$$
    Неравенство КБШ в данном случае будет вот таким:
    $$\sum\limits_{i=1}^n |x_iy_i| \leq \left(\sum\limits_{i=1}^n |x_i|^2\right)^{\cfrac{1}{2}} \cdot \left ( \sum\limits_{i=1}^{n} |y_i|^2\right)^{\cfrac{1}{2}}$$
    А неравенство треугольника у нас будет вот таким:
    $$\left(\sum\limits_{i=1}^n|x_i+y_i|^2\right)^{\cfrac{1}{2}}\leq \left(\sum\limits_{i=1}^n|x_i|^2\right) ^{\cfrac{1}{2}} + \left(\sum\limits_{i=1}^n |y_i|^2\right)^{\cfrac{1}{2}}$$
    По-другому неравентво треугольника в данном случае будет называться неравенством Минковского.
    \item Так же мы будем пользоваться вот таким примером. Пусть у нас выбран промежуток $[a,b]$ и $\integral{a}{b}|f|^2dt < \infty$. Тогда введем скалярное произведение: 
    $$(f,g) = \integral{a}{b} f \overline{g}dt$$
\end{enumerate}

\subsection{Процесс ортогонализация Грама-Шмидта. Орто-нормированный базис. Ортогональное дополнение.}

\deff{def:} Cистема ненулевых векторов $v_1,\ldots, v_m$ называется \deff{ортогональным}, если $\forall (i,j), i\neq j: (v_i,v_j) =0$.  

%todo: скипнуто замечание

\deff{def:} Система ненулевых векторов называется ортонормированной, если $v_1,\ldots, v_m: \forall (i,j): (v_i,v_j) = \delta_{ij}$. $||v_i||= 1$

\deff{Утверждение:} $v_1,\ldots, v_m$ ортогональная $\Rightarrow$ $v_1,\ldots , v_m$ \uline{линейно независимы}.

\textbf{Доказательство:}

$v_1,\ldots, v_m$ ортогональны. Хотим показать тривиальность разложения нуля(в принципе ничего нового).
$$\sum\limits_{i=1}^n \alpha_i v_i =\zero$$
Давайте применим операцию скалярного произведения с $v_j$ к обеим частям. Таким образом получим:
$$0 = \sum\limits_{i=1}^n \alpha_i (v_i,v_j)= \alpha_j (v_j,v_j) \Rightarrow \alpha_j =0$$ 
Таким образом получаю, что каждая $\alpha_j = 0$, то есть вектора линейно независимы.

\hfill Q.E.D.


\thmm{Теорема (процесс ортогонализации Грама-Шмидта)}

$\forall a_1,\ldots,a_n \rightarrow \exists b_1,\ldots,b_k \in V$. Причем $b_i$ попарно-ортогонально и $\span (a_1,\ldots, a_n) = \span(b_1,\ldots,b_k)$. При этом $k = \rg (a_1,\ldots,a_m)$

\textbf{Доказательство:}

Пусть $a_1,\ldots,a_m$ линейно независимы, то есть $m=k$

Будем доказывать по индукции.

\textbf{База:}

Рассмотрим $k=2$, пусть $b_1 = a_1$. Мы хотим, чтобы $b_2$ и $b_1$ были ортогональны, то есть $(b_1,b_2) = 0$. Пусть $b_2 = a_2 - c_1 b_1$. Тогда, нам надо, чтобы было выполнено:
$$0 = (b_2,b_1) = (a_1,b_1)-c_1(b_1,b_1) \Leftrightarrow c_1 = \cfrac{(a_2,b_1)}{(b_1,b_1)}$$
Заметим, что сейчас мы нашли такое $b_2$, что оно ортогонально и тк мы сделали линейное преобразование, то $\span (b_1,b_2) = \span (a_1,a_2)$

\textbf{Индукционный переход:}

Пусть верно для $m$. Докажем для $m+1$.

Возьму первые $m$. Для них по предположению индукцию построю ортогональные.

Возьму $b_{m+1}=a_{m+1} - \sum\limits_{j=1}^m c_j b_j$

Хотим, понять существуют ли такие $c$. Давайте переберем $r = 1\ldots m $ и возьмем скалярное произведение с $b_r$. Тогда:
$$0 = (b_{m+1}, b_r) = (a_{m+1},b_r)-\sum\limits_{j=1}^m c_j(b_j,b_r) = (a_{m+1},b_r)-c_r(b_r,b_r)$$
$c_r = \cfrac{(a_{m+1}, b_r)}{||b_r||^2}$

Заметим, что $b_{m+1}\neq 0$, тк иначе $a_{m+1}\in \span(b_1,\ldots b_n)$, откуда линейно-независимый. Откуда получили $m+1$ ортогональный.

\hfill Q.E.D.



