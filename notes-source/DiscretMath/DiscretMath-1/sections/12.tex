
\subsection{Разбиение}
\subsubsection{Разбиение на множества}
Разбиение любого множества на набор множеств, так, что каждый элемент исходного множества оказывается ровно в одном множестве набора. 

\subsubsection{Разбиение на слагаемые}
Разбиение числа на множество чисел, так, что сумма этих чисел равна самому числу.

\textit{nое} \textbf{числа Белла} $B_n$ - количество способов разбить на множества множество из n эелементов. 

\textbf{Число Стирлинга 2ого рода} из n по k - $\{_k^n\}=S_2(n, k)$

Формула числа Стирлинга 2ого рода: $$\{_k^n\} = \{_{k - 1}^{n - 1}\}+k*\{_k^{n-1}\}$$
Чтобы это доказать посмотрим на какой-то элемент из $n$, если его удалить и он был один в множестве, то кол-во множеств и элементов уменьшилось и это $\{_{k - 1}^{n - 1}\}$, если же он лежит с кем-то в множестве, то при его удалении мы получим одну и ту же конфигурацию $k$ раз, значит их можно посчитать как $k*\{_k^{n-1}\}$.

Число Белла тогда можно посчитать как: $$B_n = \sum\limits_{k=0}^n\{_k^n\}$$

Число Стирлинга 1ого рода $S_1(n, k)$ - число способов разбить множество мощностью $n$ на $k$ циклов, где цикл - размещения в котором два элемента считаются одинаковыми, если являются циклическими сдвигами друг друга. 

Формула числа Стирлинга 1ого рода: $$\left[_k^n\right] = \left[_{k-1}^{n-1}\right] + (n-1)*[^{n-1}_k]$$
Доказательство этого аналогично доказательству формулы для числа Стирлинга 2ого рода.

Пока что не ясно зачем это, но было обращено внимание на последовательности 1 7 6 1 для чисел Стирлинга 2ого рода, и 6 11 6 1 для чисел Стирлинга 1ого рода, если вы их видите, то с большой вероятностью вы видите числа Стирлинга.

Свойства чисел Стирлинга 1ого рода - $\sum\limits_{k=0}^n[_k^n] = n!$ 
\subsection{Пути Дика}
Рассмотрим плоскость и путь в нем берущий начало в точке (0,0), выполняющий следующие требования:
\begin{enumerate}
    \item [1.] Каждое звено которого направлено вправо на 1 и либо на 1 вверх, либо на 1 вниз.
    \item [2.] Путь Дика обязательно заканчивается в точке с Y-координатой равной нулю.
    \item [3.] Y-координата любой вершины в пути Дика никогда не становится отрицательной.
\end{enumerate}

Проще всего провести аналогию с графиком акций криптовалюты, они обе начинаются с нуля, растут или падают, но в конце концов всегда обесцениваются и оказываются в 0, но при этом не могут быть отрицательны ни в какое время.  

\subsection{Пентагональная формула Эйлера.}

Не будем вдаваться тут особо в подробности о том, что это такое, я скажу так, вас об этом вряд ли спросят, а уж запоминать вы это точно не захотите(Станок мельком что-то об этом упомянул и это скорее та самая легендарная \href{https://dev.mccme.ru/~merzon/mirror/mathtabletalks/files/pentagonal%20(1).pdf}{дополнительная литература по дм}).

\subsection{Бинарные деревья.}
\textbf{Бинарные деревья} - корневые деревья, ограниченные тем, что у каждой вершины не больше двух сыновей, один из которых считается правым, а второй левым. Жесткого ограничения на степень вершины нет, причем четкого предписания, что если сын один, то он обязательно левый или правый тоже нет, поэтому, например, одинаковые по строению деревья, после предписания какие сыновья левые, а какие правые, могут получится разными.

\textbf{Утверждение:} Количество бинарных деревьев на n вершинах -  $C_n$($n$-oe число Каталана)

\textbf{Доказательство:}
\begin{enumerate}
    \item Проведем аналогию с псп, для которого уже все доказано. Пусть вершина - пара открывающейся и закрывающейся скобки, ее левый сын, это та псп, что находится внутри этой пары скобок, а правый сын, та псп, что находится правее нее. Получаем, что из любого дерева мы можем однозначно получить какое-то псп, а также из любой псп можем однозначным образом получить бинарное дерево(если неясно почему, докажите сами), значит это биекция, а значит утверждаемое верно. 
\end{enumerate}

