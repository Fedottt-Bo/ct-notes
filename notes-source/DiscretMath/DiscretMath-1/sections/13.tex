
\subsection{Перестановки(глубже)}
\subsubsection{Композиция перестановок}
Перестановки можно переменожать, как это было названо на лекции, но здесь далее я буду использовать композицию, потому что это подходит по смыслу больше. При композиции перестановок $a$ и $b$ мы просто переставляем массив по перестановке $a$, а потом по перестановке $b$. Записывается это в обратном порядке - $ba$, а также всегда равно какой-то перестановке $c$, для которой верно: $c[i] = b[a[i]]$, так проще всего и запомнить правильный порядок записи $ba$.

Перестановки \textbf{ассоциативны}, доказывается это так, посмотрим куда попадет $i$ый элемент перестановки: $a(bc)[i]=a[b[c[i]]]$ и $(ab)c[i] = a[b[c[i]]]$.

У перестановок есть \textbf{нулевой элемент}, называемая стабильной перестановкой -  $1, 2, \dots, n$

Также существует и \textbf{обратный элемент}, для перестановок это та перестановка, при композиции с которой наша исходная перестановка $P$ станет стабильной. Доказательство существования такой перестановки может показаться неочевидным, но на самом деле все станет кристально понятно, если мы представим перестановку в виде \href{https://ru.wikipedia.org/wiki/%D0%9E%D1%80%D0%B8%D0%B5%D0%BD%D1%82%D0%B8%D1%80%D0%BE%D0%B2%D0%B0%D0%BD%D0%BD%D1%8B%D0%B9_%D0%B3%D1%80%D0%B0%D1%84}{ориентированного графа}, это представление, кстати, самое удобное, так вот в нем становится понятно, что перестановка это просто набор циклов, и если все ребра в нем повернуть в обратную сторону, то мы также получим какую-то перестановку, которая при композиции с исходной и будет давать стабильную, эту перестановку мы и назовем обратной.

\subsection{Циклические классы}
Мы уже забежали вперед в прошлой секции и представили перестановку в виде графа, так вот \textbf{циклические классы} - это множество циклов данной перестановки, если представить ее в виде графа. Для каждого класса оттуда можно задать его \textbf{мощность} - это просто длина цикла.

\textbf{Инволюция} - перестановка, в которой мощности циклических классов не превышает 2.

\textbf{Порядок перестановки} $P$ - степень $k$, при возведении $P$ в которую у нас получится стабильная перестановка.

Давайте научимся находить порядок перестановки $P$. Представим ее в виде графа и вспомним, что это все циклы какой-то длины. При композиции текущего массива с P мы, по факту, из каждой вершины ходим по ребру, который ведет из этой вершины в перестановке. А так как мы уже помним, что $P$ - это набор циклов каких-то длин, то нам нужно, чтобы итоговое число перемещений по ребрам было кратно всем мощностям циклов, то есть НОК всех мощностей циклов.

\textbf{Важный факт:} Перестановку можно хранить как матрицу, но никто так не делает, так что идем дальше.

\subsection{Теорема Кэли о конечных группах} 
Любая конечная группа $H$ вложена в $S_n$.

\textbf{Уточнение:} $S_n$ - множество всех перестановок из $n$ элементов.

Занумеруем элементы группы $H$, а дальше построим таблицу умножения для этой группы. Теперь сопоставим каждому элементу свою перестановку из $n$ элементов - строку из данной таблицы умножения, соответствующую данному элемент(так как в группе при умножении на разные элементы мы будем получать разные результаты, то это действительно перестановка). Сопоставим элементу $a$ перестановку $L_a$, тогда знаем мы такое: $L_{ab}h=(ab)h=a(bh)=L_aL_bh$, значит $L_{ab}=L_aL_b$.

