
\subsection{Подсчет с точностью до действия группы}
На группе заводятся действия, которые удовлетворяют следующим аксиомам:
\begin{enumerate}
    \item[1)] $h(g(x))=(h*g)(x)$, набор действий это группа и для них определено умножение
    \item[2)] $ex=x$, существует нейтральный элемент
\end{enumerate}
Для действия может быть \textbf{неподвижная точка}, для которой верно:$$I_g=\{x|x : gx=x\}$$
Для элемента $x$ существует \textbf{стабилизатор} - действия, которые его не изменяют:$$St_x = \{g|gx=x\}$$
Мы получили отношения эквивалентности, это можно проверить, проверив рефлексивность транзитивность и симметричность.
$X/G$ - множество классов эквивалентности, называемая \textbf{орбитами}.
\subsection{Лемма Бёрнсайда}
Формула - $$\displaystyle\left|X/G\right|=\frac{\sum\limits_{g\in G}|I_g|}{|G|}$$ Выглядит это буквально как среднее арифметическое размеров групп, им оно на самом деле и является.

Доказательство:

Нарисуем таблицу всех действий на всевозможные объекты. Рассмотрим какой-нибудь столбец элемента $x_i$, это будет его орбита, из-за отношения эквивалентности. Некоторые орбиты могут оказаться одинаковыми - для эквивалентных элементов, если будем находить несколько таких, то выкинем повторяющиеся. Столбцов тогда останется столько, сколько у нас всего орбит - $|X/G|$. Строк - $|G|$. А каждый элемент встречается теперь всего один раз, в том столбце, на которой орбите он лежит, но лежит он там столько раз, сколько действий превращают его в самого себя, то есть $|St_x|$ раз. Тогда мы получили такую формулу: $|X/G|*|G|=\sum\limits_x{St_x}$. Ну а эта формула уже очевидно приводится к нужной, так как сумма мощностей неподвижных точек по всем $g$ равна сумме стабилизаторов по всем $x$.  
\subsection{Теорема Пойя}
Число орбит - $|X/G| = \frac{1}{|G|}\sum\limits_g{S^{C(G)}}$ или $|X/G| = \frac{1}{|G|}\sum\limits_{i=1}^n{k_i*S^{i}}$, где $k_i$ - количество действий, перестановка которых имеет k циклов. 
Тут также доказательство я лучше вынесу в удобную \href{https://neerc.ifmo.ru/wiki/index.php?title=%D0%9B%D0%B5%D0%BC%D0%BC%D0%B0_%D0%91%D1%91%D1%80%D0%BD%D1%81%D0%B0%D0%B9%D0%B4%D0%B0_%D0%B8_%D0%A2%D0%B5%D0%BE%D1%80%D0%B5%D0%BC%D0%B0_%D0%9F%D0%BE%D0%B9%D0%B0}{ссылочку}.
